% ======================================================================
% 导言区 (Preamble)
% ======================================================================
\documentclass[11pt, a4paper]{article}

% --- 页面边距设置 ---
\usepackage[a4paper, top=2.5cm, bottom=2.5cm, left=2cm, right=2cm]{geometry}

% --- 字体与语言设置 (核心) ---
\usepackage{fontspec} % 允许使用系统字体

% 配置verbatim环境使用支持Unicode的等宽字体
% 设置等宽字体为支持Unicode的字体(fontspec会自动查找系统字体)
% Windows系统优先使用Consolas,如果不可用则自动回退到其他支持Unicode的等宽字体
\setmonofont{Consolas}[Scale=0.9]

% 使用ctex包,并设置默认字体为FangSong
% 通过PassOptionsToPackage在加载ctex之前禁用默认字体设置,避免"Redefining CJKfamily"警告
\PassOptionsToPackage{fontset=none}{ctex}
\usepackage{ctex}

% 设置CJK字体(在ctex加载后统一设置)
% 修复xeCJK的verbatim字体缩放警告和FangSong字体斜体形状未定义的警告
% 设置CJK等宽字体的缩放比例,解决xeCJKVerbAddon警告
% 根据警告信息,设置Scale=0.9896到CJKfamily \CJKttdefault
% 同时设置斜体形状为正常形状,避免"Font shape `TU/FangSong(0)/m/it' undefined"和"Font shape `TU/FangSong(1)/m/it' undefined"警告
\setCJKmainfont{FangSong}[ItalicFont=FangSong]
\setCJKsansfont{FangSong}[ItalicFont=FangSong]
\setCJKmonofont{FangSong}[Scale=0.9896, ItalicFont=FangSong]

% 抑制xeCJK的"Redefining CJKfamily"警告(这些警告是信息性的,不影响编译)
% 如果仍然看到警告,可以取消下面的注释来完全抑制
% \makeatletter
% \def\@xeCJK@warning#1{}
% \makeatother

% 如果Consolas不可用,可以尝试以下字体之一(取消注释):
% \setmonofont{Courier New}[Scale=0.9]
% \setmonofont{DejaVu Sans Mono}[Scale=0.9]
% \setmonofont{SimSun}[Scale=0.9]  % 宋体也支持这些Unicode字符

% --- TOC 深度设置 (新增或修改) ---
\setcounter{tocdepth}{3} % 目录深度到 subsubsection(显示到三级标题)
\setcounter{secnumdepth}{3} % 编号深度到 subsubsection(编号到三级标题)

% --- 目录格式设置 ---
% 设置目录中章节标题的字体大小和间距
\makeatletter
\renewcommand{\@pnumwidth}{2.5em}  % 页码宽度
\renewcommand{\@tocrmarg}{3.5em}   % 右边界距
\renewcommand{\@dotsep}{4.5}        % 点线间距
\makeatother

% 修正非英语语言环境下的列表标签
\usepackage{enumitem}
\setlist[itemize]{label=-}

% --- 其他常用宏包 ---
\usepackage{amsmath} % 数学公式
\usepackage{booktabs} % 用于 \addlinespace,改善表格线条
\usepackage{graphicx} % 图像支持 (封面占位符需要)
\usepackage{indentfirst}

% 定义用于伪代码和代码块的样式
\usepackage{listings}
% 使用 listings 宏包定义一个用于伪代码的环境
\lstset{
    basicstyle=\small\ttfamily,
    columns=fullflexible,
    breaklines=true,
    frame=single,
    framerule=0pt,
}

% 配置verbatim环境使用支持Unicode的字体
% 重新定义verbatim环境以使用配置的等宽字体
\usepackage{fancyvrb}
\fvset{fontsize=\small}

% 定义伪代码环境:将所有的 \lstset 参数都放在这里
\lstnewenvironment{pseudocode}[1][]{
    \lstset{
        language=Mathematica, % 使用一个不常用的语言来禁用语法高亮
        % 将任何额外的设置参数通过 #1 传递
        #1
    }
}{}

% ---------------------------
% 使用 titletoc 美化目录(必须在hyperref之前)
% ---------------------------
\usepackage{titletoc}

% hyperref 必须在最后加载(在titletoc之后)
\usepackage{hyperref}
\hypersetup{
    colorlinks=true,
    linkcolor=blue,
    filecolor=magenta,
    urlcolor=cyan,
    pdftitle={编译器构造报告},
}

% section - 一级标题格式
\titlecontents{section}
  [0em]                      % 左缩进
  {\vspace{0.4em}\bfseries}  % 之前的间距和格式(加粗)
  {\contentslabel{2.5em}}     % 章节编号宽度和格式
  {}                          % 无编号时的格式(空,使用默认)
  {\titlerule*[0.8em]{.}\contentspage}  % 点线和页码
  [\vspace{0.2em}]            % 之后的间距

% subsection - 二级标题格式
\titlecontents{subsection}
  [2em]                       % 左缩进
  {\vspace{0.3em}}            % 之前的间距
  {\contentslabel{3em}}       % 章节编号宽度和格式
  {}                          % 无编号时的格式(空,使用默认)
  {\titlerule*[0.8em]{.}\contentspage}  % 点线和页码
  [\vspace{0.15em}]           % 之后的间距

% subsubsection - 三级标题格式
\titlecontents{subsubsection}
  [4.5em]                     % 左缩进
  {\vspace{0.2em}}            % 之前的间距
  {\contentslabel{4em}}       % 章节编号宽度和格式
  {}                          % 无编号时的格式(空,使用默认)
  {\titlerule*[0.8em]{.}\contentspage}  % 点线和页码
  [\vspace{0.1em}]            % 之后的间距

% 包含个人信息定义(必须在\begin{document}之前)
% personal_info.tex
% 定义小组及个人信息变量

% --- 公共信息 (假设大家是同一个专业/班级) ---
\newcommand{\StudentMajor}{软件工程}
\newcommand{\StudentGrade}{2023级}
\newcommand{\StudentClass}{软工2班}

% --- 小组成员信息 (4人) ---
% 成员 1
\newcommand{\NameA}{陈添硕}
\newcommand{\IdA}{3023244225}
\newcommand{\WorkA}{语法分析器设计与实现}
\newcommand{\ReflectionA}{在实现语法分析器的过程中,深入理解了自顶向下和自底向上两种分析方法的区别与联系。通过实现递归下降分析器,掌握了如何将语法规则转化为递归函数,理解了向前看(lookahead)在解决选择冲突中的重要作用。在构建SLR分析表时,通过计算FIRST集和FOLLOW集,深刻体会到了形式化方法在编译器设计中的严谨性。语法分析器的实现让我认识到,良好的设计需要充分考虑语法的特点和实际应用场景。}

% 成员 2
\newcommand{\NameB}{李昊蓬}
\newcommand{\IdB}{3023244162}
\newcommand{\WorkB}{IR生成器设计和实现}
\newcommand{\ReflectionB}{IR生成是编译器中最核心也最复杂的部分。通过实现访问者模式遍历AST并生成LLVM IR,深入理解了中间代码生成的过程。在处理类型提升、常量折叠、短路求值等特性时,深刻体会到了编译器优化的复杂性。符号表的作用域管理是一个重要挑战,通过实现作用域栈成功解决了变量可见性问题。控制流图的生成和Phi节点的使用让我理解了SSA形式的重要性。这次经历让我对编译器的后端有了更深入的认识。}

% 成员 3
\newcommand{\NameC}{许英帅}
\newcommand{\IdC}{3023244221}
\newcommand{\WorkC}{词法分析器设计和实现}
\newcommand{\ReflectionC}{通过实现词法分析器,深入理解了词法分析的基本原理。在实现DFA状态机的过程中,掌握了如何将正则表达式转化为确定有限自动机,理解了状态转换表的设计方法。关键字识别、标识符扫描、数字解析和注释处理等细节的实现让我认识到词法分析的复杂性。通过实现NFA到DFA的转换,进一步理解了形式语言理论在实际应用中的价值。词法分析器作为编译器的第一道关卡,其正确性对整个编译过程至关重要。}

% 成员 4
\newcommand{\NameD}{谢雨航}
\newcommand{\IdD}{3023244222}
\newcommand{\WorkD}{测试与调试,文档编写}
\newcommand{\ReflectionD}{在测试与调试过程中,学会了如何系统地验证编译器的正确性。通过设计测试用例覆盖各种边界情况,发现了许多隐藏的bug。调试过程让我深刻理解了编译器各个模块之间的依赖关系,学会了如何定位问题所在。文档编写工作让我对整个编译器的架构有了全面的认识,通过整理和总结,不仅帮助了团队协作,也加深了自己对编译器原理的理解。这次经历让我认识到,测试和文档是软件开发中不可或缺的重要环节。}


% ======================================================================
% 文档区 (Document)
% ======================================================================
\begin{document}

% ----------------------------------------------------------------------
% 封面页 (Title Page)
% ----------------------------------------------------------------------
\begin{titlepage}
    \centering % 整体居中

    % 1. 校徽
    \vspace*{2cm} % 距离顶部一些空间

    \includegraphics[width=0.5\linewidth]{logo.png}

    \vfill % 自动填充垂直空间

    % 2. 报告名称
    {\Huge C--编译器源码结构和功能详细说明}

    \vspace{1cm}

    {\Large 编译器构造报告}

    \vfill % 自动填充垂直空间

	% 3. 小组信息展示 (四人信息格式)
    % 调整了 makebox 的宽度 (从5cm改为7cm) 以容纳"姓名+学号"
    \begin{tabular}{rl}
        \textbf{专    业:} & \underline{\makebox[7cm][c]{\StudentMajor}} \\
        \addlinespace[10pt]
        \textbf{年    级:} & \underline{\makebox[7cm][c]{\StudentGrade}} \\
        \addlinespace[10pt]
        \textbf{班    级:} & \underline{\makebox[7cm][c]{\StudentClass}} \\
        \addlinespace[10pt]

        % 成员列表
        \textbf{组    员:} & \underline{\makebox[7cm][c]{\NameA \quad (\IdA)}} \\
        \addlinespace[8pt] % 调整行间距
                            & \underline{\makebox[7cm][c]{\NameB \quad (\IdB)}} \\
        \addlinespace[8pt]
                            & \underline{\makebox[7cm][c]{\NameC \quad (\IdC)}} \\
        \addlinespace[8pt]
                            & \underline{\makebox[7cm][c]{\NameD \quad (\IdD)}} \\
    \end{tabular}

    \vspace{3cm} % 距离底部一些空间
\end{titlepage}

% --- 目录页 ---

\clearpage

% 目录标题区域
\begin{center}
    \vspace*{1cm}
    {\Huge \textbf{目 录}}
    \vspace{0.8cm}
\end{center}

% 生成目录(自动生成所有章节)
% 注意:首次编译可能不显示目录,需要编译两次才能生成完整的目录
\tableofcontents

\clearpage

\setcounter{page}{1}  % 从正文开始重新计数页码
\pagestyle{plain}     % 设置正文页码样式


% --- 编译器源码结构和功能详细说明 ---

\section{词法分析器}

\subsection{算法描述}

\subsubsection{基于NFA→DFA转换的词法分析器 (\texttt{SLRLexer})}

\paragraph{算法流程:}
\begin{enumerate}
    \item \textbf{构建NFA} (\texttt{buildDFA()} 函数):
    \begin{itemize}
        \item 为关键字、标识符、整数、浮点数、运算符、界符分别构建NFA
        \item 使用epsilon转换合并所有NFA为一个组合NFA
        \item 通过子集构造法(Subset Construction)将NFA转换为DFA
        \item 使用DFA最小化算法优化DFA
    \end{itemize}

    \item \textbf{最长匹配算法} (\texttt{analyze()} 函数):
    \begin{pseudocode}
初始化: currentState = dfa.start, startPos = pos

WHILE pos < source.length():
    跳过空白字符和注释

    currentState = dfa.start
    startPos = pos
    lastAcceptState = nullptr
    lastAcceptPos = -1
    currentPos = pos

    // 最长匹配
    WHILE currentPos < source.length():
        char c = sourceCode[currentPos]
        IF currentState有c的转换:
            currentState = currentState.transitions[c]
            currentPos++
            IF currentState是接受状态:
                lastAcceptState = currentState
                lastAcceptPos = currentPos
        ELSE:
            BREAK
    END WHILE

    IF lastAcceptState != nullptr AND lastAcceptPos > startPos:
        创建token: Token(lastAcceptState->acceptType,
                       sourceCode.substr(startPos, lastAcceptPos - startPos))
        pos = lastAcceptPos
    ELSE:
        创建ERROR token
        pos++
END WHILE

添加EOF token
    \end{pseudocode}
\end{enumerate}

\subsection{输出格式说明}

\textbf{Token输出格式:} \texttt{[单词符号] TAB <[类型],[属性]>}

\textbf{类型分类:}
\begin{itemize}
    \item \textbf{KW (关键字)}: int(1), void(2), return(3), const(4), float(6), if(7), else(8)
    \item \textbf{OP (运算符)}: +(30), -(31), *(32), /(33), \%(34), =(35), >(36), <(37), ==(38), <=(39), >=(40), !=(41), \&\&(42), ||(43), !(44)
    \item \textbf{SE (界符)}: ((50), )(51), \{(52), \}(53), ;(54), ,(55)
    \item \textbf{IDN (标识符)}: \texttt{<IDN,标识符名称>}
    \item \textbf{INT (整数)}: \texttt{<INT,整数值>}
    \item \textbf{FLOAT (浮点数)}: \texttt{<FLOAT,浮点数值>}
    \item \textbf{EOF (文件结束)}: \texttt{\$}
\end{itemize}

\paragraph{示例输出:}
\begin{verbatim}
int	<KW,1>
main	<IDN,main>
(	<SE,50>
)	<SE,51>
{	<SE,52>
return	<KW,3>
10	<INT,10>
;	<SE,54>
}	<SE,53>
\end{verbatim}

\subsection{代码实现细节}

\subsubsection{Token结构定义}

\textbf{Token结构体:}
\begin{lstlisting}[language=C++]
struct Token {
    TokenType type;      // Token类型
    std::string value;   // Token值
    int line;            // 行号
    int column;          // 列号

    // 方法:获取类型字符串表示、输出格式化Token等
    std::string getTypeString() const;
    std::string toString() const;
};
\end{lstlisting}

\subsubsection{关键字表初始化}

\textbf{功能:}初始化关键字映射表,将关键字字符串映射到对应的TokenType。

\textbf{函数声明:}
\begin{lstlisting}[language=C++]
void Lexer::initKeywords();
\end{lstlisting}

\textbf{功能说明:}建立关键字(int, void, return, const, float, if, else)到TokenType的映射关系,关键字不区分大小写。

\subsubsection{标识符扫描实现}

\textbf{功能:}扫描标识符或关键字。

\textbf{函数声明:}
\begin{lstlisting}[language=C++]
Token Lexer::scanIdentifier();
\end{lstlisting}

\textbf{功能说明:}扫描字母、数字和下划线组成的标识符,然后检查是否是关键字(不区分大小写),如果是关键字则返回对应的关键字Token,否则返回标识符Token。

\subsubsection{数字扫描实现}

\textbf{功能:}扫描整数或浮点数。

\textbf{函数声明:}
\begin{lstlisting}[language=C++]
Token Lexer::scanNumber();
\end{lstlisting}

\textbf{功能说明:}扫描数字序列,如果遇到小数点且后面有数字,则识别为浮点数,否则识别为整数。

\subsubsection{注释处理实现}

\textbf{功能:}跳过行注释(//)或块注释(/* */)。

\textbf{函数声明:}
\begin{lstlisting}[language=C++]
bool Lexer::skipComment();
\end{lstlisting}

\textbf{功能说明:}检测并跳过注释。行注释跳过到行尾,块注释跳过到结束标记\texttt{*/}。如果块注释未闭合则报错。

\subsection{源程序编译步骤}

\textbf{使用词法分析器的编译步骤:}

\begin{enumerate}
    \item \textbf{加载源文件}:
    \begin{lstlisting}[language=C++]
Lexer lexer;
lexer.loadFromFile("source.sy");
    \end{lstlisting}

    \item \textbf{执行词法分析}:
    \begin{lstlisting}[language=C++]
lexer.tokenize();
    \end{lstlisting}

    \item \textbf{获取Token序列}:
    \begin{lstlisting}[language=C++]
const auto& tokens = lexer.getTokens();
    \end{lstlisting}

    \item \textbf{输出结果}:
    \begin{lstlisting}[language=C++]
lexer.printTokens();  // 打印到标准输出
// 或
std::string tokenStr = lexer.getTokensString();  // 获取字符串
    \end{lstlisting}
\end{enumerate}

\textbf{命令行使用:}
\begin{verbatim}
./compiler -l source.sy    # 仅词法分析
\end{verbatim}

\clearpage
\section{语法分析器}

\subsection{算法描述}

本项目实现了两种语法分析器:

\subsubsection{方法一:SLR语法分析器 (\texttt{SLRParser})} % 对应源码结构文档的方法一

\paragraph{算法流程:}
\begin{enumerate}
    \item \textbf{初始化文法} (\texttt{initGrammar()}):
    \begin{itemize}
        \item 定义81个产生式规则
        \item 识别终结符和非终结符集合
    \end{itemize}
    \item \textbf{计算FIRST集} (\texttt{computeFirst()}):
    \begin{pseudocode}
FOR 每个终结符t:
    FIRST(t) = {t}

WHILE FIRST集发生变化:
    FOR 每个产生式 A -> α:
        IF α = ε:
            FIRST(A) += {ε}
        ELSE:
            FOR 每个符号X in α:
                FIRST(A) += FIRST(X) - {ε}
                IF ε NOT IN FIRST(X):
                    BREAK
                IF 到达α的末尾:
                    FIRST(A) += {ε}
    END WHILE
    \end{pseudocode}

    \item \textbf{计算FOLLOW集} (\texttt{computeFollow()}):
    \begin{pseudocode}
FOLLOW(开始符号) = {$}

WHILE FOLLOW集发生变化:
    FOR 每个产生式 A -> αBβ:
        IF β存在:
            FOLLOW(B) += FIRST(β) - {ε}
            IF ε IN FIRST(β):
                FOLLOW(B) += FOLLOW(A)
        ELSE:
            FOLLOW(B) += FOLLOW(A)
    END WHILE
    \end{pseudocode}

    \item \textbf{构建LR(0)项目集族} (\texttt{buildCollection()}):
    \begin{pseudocode}
I0 = closure({S' -> ·Program})
canonicalCollection = {I0}

WHILE 有新的项目集产生:
    FOR 每个项目集I:
        FOR 每个符号X (终结符或非终结符):
            J = goto(I, X)
            IF J非空 AND J不在canonicalCollection中:
                将J添加到canonicalCollection
                记录转换: (I, X) -> J
    END WHILE
    \end{pseudocode}

    \item \textbf{构建SLR分析表} (\texttt{buildTable()}):
    \begin{pseudocode}
FOR 每个项目集Ii:
    FOR 每个项目 [A -> α·aβ] in Ii:
        J = goto(Ii, a)
        action[i, a] = shift j

    FOR 每个项目 [A -> α·] in Ii:
        IF A != S':
            FOR 每个a IN FOLLOW(A):
                action[i, a] = reduce A -> α
        ELSE:
            action[i, $] = accept

    FOR 每个非终结符A:
        IF goto(Ii, A) = Ij:
            goto[i, A] = j
    \end{pseudocode}

    \item \textbf{SLR分析算法} (\texttt{parse()}):
    \begin{pseudocode}
初始化: stateStack = [0], valueStack = []
ip = 0  // 输入指针

WHILE true:
    s = stateStack.top()
    a = tokens[ip]  // 当前输入符号

    IF action[s, a] = shift t:
        stateStack.push(t)
        valueStack.push(语义值)
        ip++

    ELSE IF action[s, a] = reduce A -> β:
        从栈中弹出|β|个状态和语义值
        t = stateStack.top()
        stateStack.push(goto[t, A])
        valueStack.push(reduce(产生式ID, 语义值列表))

    ELSE IF action[s, a] = accept:
        RETURN success

    ELSE:
        RETURN error
END WHILE
    \end{pseudocode}
\end{enumerate}

\subsubsection{方法二:递归下降语法分析器 (\texttt{Parser})}

\paragraph{算法特点:}
\begin{itemize}
    \item 为每个非终结符编写一个递归函数
    \item 通过向前看(lookahead)解决选择冲突
    \item 自顶向下构建AST
\end{itemize}

\textbf{主要分析函数:}
\begin{itemize}
    \item \texttt{parseCompUnit()}: 解析编译单元
    \item \texttt{parseConstDecl()}: 解析常量声明
    \item \texttt{parseVarDecl()}: 解析变量声明
    \item \texttt{parseFuncDef()}: 解析函数定义
    \item \texttt{parseStmt()}: 解析语句
    \item \texttt{parseExp()}: 解析表达式(通过parseAddExp)
    \item \texttt{parseCond()}: 解析条件表达式
\end{itemize}

\textbf{算法流程示例(parseAddExp):}
\begin{pseudocode}
parseAddExp():
    left = parseMulExp()

    WHILE 当前token是 + 或 -:
        保存运算符op
        前进token
        right = parseMulExp()
        创建新的AddExpNode,left作为左子树,right作为右子树
        left = 新节点
    END WHILE

    RETURN left
\end{pseudocode}

\subsection{SLR分析表结构} % 新增子节,对应源码文档 2.2

分析表包含两部分:
\begin{enumerate}
    \item \textbf{ACTION表}:$\texttt{actionTable}[\text{state}, \text{terminal}]$ $\rightarrow \text{Action}$
    \begin{itemize}
        \item \texttt{SHIFT n}: 移进,跳转到状态n
        \item \texttt{REDUCE m}: 归约,使用产生式m
        \item \texttt{ACC}: 接受
        \item \texttt{ERR}: 错误
    \end{itemize}
    \item \textbf{GOTO表}:$\texttt{gotoTable}[\text{state}, \text{nonTerminal}]$ $\rightarrow \text{state}$
    \begin{itemize}
        \item 记录状态转换
    \end{itemize}
\end{enumerate}

\paragraph{冲突解决:}
\begin{itemize}
    \item 移进-归约冲突:优先移进(shift-reduce conflict resolution)
\end{itemize}

\paragraph{示例分析表片段:}
\begin{verbatim}
State 0:
  action[0, "int"] = shift 5
  action[0, "const"] = shift 6
  goto[0, "Program"] = 1
  goto[0, "compUnit"] = 2

State 5:
  action[5, "Ident"] = shift 10
  goto[5, "bType"] = 7

State 10:
  action[10, "="] = shift 15
  action[10, ";"] = reduce 19  // varDef -> Ident
  ...
\end{verbatim}

\subsection{输出格式说明}

\textbf{递归下降分析器输出格式:}
\begin{verbatim}
步骤号	TAB	Token值	TAB	动作
1	int	move
2	main	move
3	(	move
4	)	move
5	{	move
6	return	move
7	10	move
8	;	move
9	Stmt	reduction
10	Block	reduction
11	FuncDef	reduction
12	compUnit	reduction
13	$#$	accept
\end{verbatim}

\textbf{动作类型:}
\begin{itemize}
    \item \texttt{move}: 移进(匹配终结符)
    \item \texttt{reduction}: 归约(应用产生式)
    \item \texttt{accept}: 接受
    \item \texttt{error}: 错误
\end{itemize}

\subsection{代码实现细节}

\subsubsection{递归下降分析器核心实现}

\textbf{Token匹配和前进:}
\begin{lstlisting}[language=C++]
Token Parser::currentToken() const;  // 获取当前Token
bool Parser::match(TokenType type);  // 匹配并前进Token
bool Parser::expect(TokenType type, const std::string& msg);  // 期望匹配Token,失败则报错
\end{lstlisting}

\textbf{功能说明:}
\begin{itemize}
    \item \texttt{currentToken()}: 返回当前待处理的Token,如果已到文件末尾则返回EOF Token
    \item \texttt{match()}: 如果当前Token类型匹配,则前进到下一个Token并返回true,否则返回false
    \item \texttt{expect()}: 期望匹配指定类型的Token,如果失败则输出错误信息
\end{itemize}

\textbf{编译单元解析:}
\begin{lstlisting}[language=C++]
std::shared_ptr<CompUnitNode> Parser::parseCompUnit();
\end{lstlisting}

\textbf{功能说明:}解析编译单元,循环处理全局声明和函数定义。通过向前看(lookahead)区分函数定义和变量声明。

\textbf{表达式解析(左结合):}
\begin{lstlisting}[language=C++]
std::shared_ptr<AddExpNode> Parser::parseAddExp();
\end{lstlisting}

\textbf{功能说明:}解析加法表达式,处理左结合的\texttt{+}和\texttt{-}运算符。先解析第一个mulExp作为左操作数,然后循环处理后续的运算符和右操作数,构建左结合的AST。

\subsubsection{AST节点创建示例}

\textbf{函数定义节点创建:}
\begin{lstlisting}[language=C++]
std::shared_ptr<FuncDefNode> Parser::parseFuncDef();
\end{lstlisting}

\textbf{功能说明:}解析函数定义,包括返回类型、函数名、参数列表和函数体。返回类型可以是int、float或void。参数列表可能为空。

\subsection{源程序编译步骤}

\textbf{使用语法分析器的编译步骤:}

\begin{enumerate}
    \item \textbf{词法分析}:
    \begin{lstlisting}[language=C++]
SLRLexer lexer;
auto tokens = lexer.analyze(sourceCode);
    \end{lstlisting}

    \item \textbf{语法分析}:
    \begin{lstlisting}[language=C++]
SLRParser parser;
bool success = parser.parse(tokens);
    \end{lstlisting}

    \item \textbf{获取AST}:
    \begin{lstlisting}[language=C++]
auto ast = parser.getAST();
    \end{lstlisting}
\end{enumerate}

\textbf{命令行使用:}
\begin{verbatim}
./compiler -p source.sy    # 词法+语法分析
\end{verbatim}

\clearpage
\section{语法树(AST)}

\subsection{语法树存储结构} % 对应源码文档 3.1

\paragraph{AST节点层次结构:}
\begin{verbatim}
CompUnitNode (根节点)
├── decls: vector<DeclNode*>          // 全局声明列表
│   ├── ConstDeclNode
│   │   ├── bType: BType
│   │   └── constDefs: vector<ConstDefNode*>
│   │       └── ConstDefNode
│   │           ├── ident: string
│   │           └── initVal: ExpNode*
│   └── VarDeclNode
│       ├── bType: BType
│       └── varDefs: vector<VarDefNode*>
│           └── VarDefNode
│               ├── ident: string
│               └── initVal: ExpNode* (可选)
│
└── funcDefs: vector<FuncDefNode*>     // 函数定义列表
    └── FuncDefNode
        ├── returnType: BType
        ├── ident: string
        ├── params: vector<FuncFParamNode*>
        └── block: BlockNode*
            └── items: vector<BlockItemNode*>
                ├── decl: DeclNode* (可选)
                └── stmt: StmtNode* (可选)
                    ├── ASSIGN: lVal + exp
                    ├── EXP: exp (可选)
                    ├── BLOCK: block
                    ├── IF: cond + thenStmt + elseStmt (可选)
                    └── RETURN: exp (可选)
\end{verbatim}

\paragraph{表达式节点层次:}
\begin{verbatim}
ExpNode (基类)
├── AddExpNode
│   ├── left: AddExpNode* (可选)
│   ├── op: BinaryOp (ADD/SUB)
│   └── right: MulExpNode*
│
├── MulExpNode
│   ├── left: MulExpNode* (可选)
│   ├── op: BinaryOp (MUL/DIV/MOD)
│   └── right: UnaryExpNode*
│
├── UnaryExpNode
│   ├── PRIMARY: primaryExp
│   ├── FUNC_CALL: funcName + args
│   └── UNARY_OP: unaryOp + unaryExp
│
├── PrimaryExpNode
│   ├── PAREN_EXP: exp
│   ├── LVAL: lVal
│   └── NUMBER: number
│
├── LValNode
│   └── ident: string
│
└── NumberNode
    ├── isFloat: bool
    ├── intVal: int
    └── floatVal: float
\end{verbatim}

\paragraph{存储方式:}
\begin{itemize}
    \item 使用 \texttt{std::shared\_ptr} 智能指针管理节点生命周期
    \item 树形结构通过指针链接
    \item 列表使用 \texttt{std::vector<std::shared\_ptr<Node>>}
\end{itemize}

\subsubsection{AST节点类定义}

\textbf{AST节点基类:}
\begin{lstlisting}[language=C++]
class ASTNode {
public:
    virtual ~ASTNode() = default;
    std::string type;  // 节点类型名称

    ASTNode() : type("ASTNode") {}
    explicit ASTNode(const std::string& t) : type(t) {}
};
\end{lstlisting}

\textbf{编译单元节点:}
\begin{lstlisting}[language=C++]
class CompUnitNode : public ASTNode {
public:
    std::vector<std::shared_ptr<DeclNode>> decls;        // 全局声明
    std::vector<std::shared_ptr<FuncDefNode>> funcDefs;  // 函数定义

    CompUnitNode() : ASTNode("CompUnit") {}
};
\end{lstlisting}

\textbf{声明节点:}
\begin{lstlisting}[language=C++]
class DeclNode : public ASTNode {
public:
    bool isConst = false;
    DeclNode() : ASTNode("Decl") {}
};

class ConstDeclNode : public DeclNode {
public:
    BType bType;
    std::vector<std::shared_ptr<ConstDefNode>> constDefs;
    ConstDeclNode() : DeclNode("ConstDecl") { isConst = true; }
};

class VarDeclNode : public DeclNode {
public:
    BType bType;
    std::vector<std::shared_ptr<VarDefNode>> varDefs;
    VarDeclNode() : DeclNode("VarDecl") { isConst = false; }
};
\end{lstlisting}

\textbf{函数定义节点:}
\begin{lstlisting}[language=C++]
class FuncDefNode : public ASTNode {
public:
    BType returnType;
    std::string ident;
    std::vector<std::shared_ptr<FuncFParamNode>> params;
    std::shared_ptr<BlockNode> block;

    FuncDefNode() : ASTNode("FuncDef") {}
};
\end{lstlisting}

\textbf{表达式节点层次:}
\begin{lstlisting}[language=C++]
// 表达式基类
class ExpNode : public ASTNode {
public:
    ExpNode() : ASTNode("Exp") {}
};

// 加法表达式(左结合)
class AddExpNode : public ExpNode {
public:
    std::shared_ptr<AddExpNode> left;   // 左操作数(可选)
    BinaryOp op;                        // 运算符 (ADD, SUB)
    std::shared_ptr<MulExpNode> right;  // 右操作数

    AddExpNode() : ExpNode("AddExp"), op(BinaryOp::ADD) {}
};

// 乘法表达式(左结合)
class MulExpNode : public ExpNode {
public:
    std::shared_ptr<MulExpNode> left;   // 左操作数(可选)
    BinaryOp op;                        // 运算符 (MUL, DIV, MOD)
    std::shared_ptr<UnaryExpNode> right; // 右操作数

    MulExpNode() : ExpNode("MulExp"), op(BinaryOp::MUL) {}
};

// 一元表达式
class UnaryExpNode : public ExpNode {
public:
    enum class UnaryType {
        PRIMARY,    // primaryExp
        FUNC_CALL,  // Ident '(' funcRParams? ')'
        UNARY_OP    // unaryOp unaryExp
    };

    UnaryType unaryType;
    std::shared_ptr<PrimaryExpNode> primaryExp; // PRIMARY
    std::string funcName;                        // FUNC_CALL
    std::vector<std::shared_ptr<ExpNode>> args; // FUNC_CALL
    UnaryOp unaryOp;                            // UNARY_OP
    std::shared_ptr<UnaryExpNode> unaryExp;     // UNARY_OP

    UnaryExpNode() : ExpNode("UnaryExp"), unaryType(UnaryType::PRIMARY) {}
};

// 逻辑与表达式(短路求值)
class LAndExpNode : public ExpNode {
public:
    std::shared_ptr<LAndExpNode> left;  // 左操作数(可选)
    std::shared_ptr<EqExpNode> right;   // 右操作数

    LAndExpNode() : ExpNode("LAndExp") {}
};

// 逻辑或表达式(短路求值)
class LOrExpNode : public ExpNode {
public:
    std::shared_ptr<LOrExpNode> left;    // 左操作数(可选)
    std::shared_ptr<LAndExpNode> right; // 右操作数

    LOrExpNode() : ExpNode("LOrExp") {}
};
\end{lstlisting}

\textbf{语句节点:}
\begin{lstlisting}[language=C++]
enum class StmtType {
    ASSIGN,     // lVal '=' exp ';'
    EXP,        // (exp)? ';'
    BLOCK,      // block
    IF,         // 'if' '(' cond ')' stmt ('else' stmt)?
    RETURN      // 'return' (exp)? ';'
};

class StmtNode : public ASTNode {
public:
    StmtType stmtType;

    // ASSIGN类型
    std::shared_ptr<LValNode> lVal;
    std::shared_ptr<ExpNode> exp;

    // BLOCK类型
    std::shared_ptr<BlockNode> block;

    // IF类型
    std::shared_ptr<CondNode> cond;
    std::shared_ptr<StmtNode> thenStmt;
    std::shared_ptr<StmtNode> elseStmt;  // 可选

    // RETURN类型 - 使用exp字段
    StmtNode() : ASTNode("Stmt"), stmtType(StmtType::EXP) {}
};
\end{lstlisting}

\clearpage
\section{中间代码生成(LLVM IR)}

\subsection{IR生成器整体架构} % 对应源码文档 3.3.1

\paragraph{设计模式:访问者模式(Visitor Pattern)}
IR生成器采用访问者模式遍历AST并生成LLVM IR:
\begin{itemize}
    \item 为每种AST节点类型定义专门的访问函数(如\texttt{visitCompUnit}、\texttt{visitFuncDef}等)
    \item 通过动态分发(\texttt{dynamic\_cast})识别节点类型并调用相应的访问函数
    \item 递归遍历AST树,自顶向下生成IR
\end{itemize}

\paragraph{核心组件:}
\begin{itemize}
    \item \texttt{IRGenerator}: IR生成器主类
    \item \texttt{IRBuilder}: IR指令构建器,负责创建各种IR指令
    \item \texttt{SymbolTable}: 符号表,管理作用域和变量映射
\end{itemize}

\paragraph{访问者模式遍历AST:}
\begin{pseudocode}
算法: visitCompUnit(node)
输入: CompUnitNode* node
输出: 无

BEGIN
    // 先处理全局声明
    FOR EACH decl IN node->decls:
        visitDecl(decl)
    END FOR

    // 再处理函数定义
    FOR EACH funcDef IN node->funcDefs:
        visitFuncDef(funcDef)
    END FOR
END

算法: visitDecl(node)
输入: DeclNode* node
输出: 无

BEGIN
    IF node是ConstDeclNode:
        visitConstDecl(node)
    ELSE IF node是VarDeclNode:
        visitVarDecl(node)
    END IF
END

算法: visitFuncDef(node)
输入: FuncDefNode* node
输出: 无

BEGIN
    // 创建函数
    retType = bTypeToLLVMType(node->returnType)
    paramTypes = []
    FOR EACH param IN node->params:
        paramTypes.append(bTypeToLLVMType(param->bType))
    END FOR

    func = Function::create(FunctionType::get(retType, paramTypes),
                              node->ident, module)

    // 创建入口基本块
    entryBB = BasicBlock::create(module, node->ident + "_ENTRY", func)
    builder->set_insert_point(entryBB)

    // 进入函数作用域
    symbolTable.enterScope()

    // 处理参数
    FOR EACH param IN node->params:
        alloca = createLocalVariable(param->ident, paramType)
        builder->create_store(arg, alloca)
        symbolTable.insert(param->ident, alloca, paramType, false)
    END FOR

    // 处理函数体
    visitBlock(node->block)

    // 如果基本块没有终结指令,添加默认返回
    IF currentBB没有终结指令:
        IF retType是void:
            builder->create_void_ret()
        ELSE:
            builder->create_ret(ConstantInt::get(0, module))
        END IF
    END IF

    // 退出函数作用域
    symbolTable.exitScope()
END

算法: visitStmt(node)
输入: StmtNode* node
输出: 无

BEGIN
    SWITCH node->stmtType:
        CASE ASSIGN:
            addr = visitLVal(node->lVal, false)  // 获取地址
            val = visitExp(node->exp)             // 计算值
            builder->create_store(val, addr)      // 存储
        CASE EXP:
            IF node->exp存在:
                visitExp(node->exp)               // 计算表达式(丢弃结果)
            END IF
        CASE BLOCK:
            visitBlock(node->block)
        CASE IF:
            visitIfStmt(node)
        CASE RETURN:
            visitReturnStmt(node)
    END SWITCH
END

算法: visitExp(node)
输入: ExpNode* node
输出: Value* (IR值)

BEGIN
    IF node是AddExpNode:
        RETURN visitAddExp(node)
    ELSE IF node是MulExpNode:
        RETURN visitMulExp(node)
    ELSE IF node是UnaryExpNode:
        RETURN visitUnaryExp(node)
    ELSE IF node是PrimaryExpNode:
        RETURN visitPrimaryExp(node)
    ELSE IF node是LValNode:
        RETURN visitLVal(node, true)  // 加载值
    ELSE IF node是NumberNode:
        RETURN visitNumber(node)
    END IF
END

算法: visitAddExp(node)
输入: AddExpNode* node
输出: Value*

BEGIN
    IF node->left == nullptr:
        // 只有右操作数
        RETURN visitMulExp(node->right)
    END IF

    // 递归计算左操作数
    left = visitAddExp(node->left)
    right = visitMulExp(node->right)

    // 类型提升
    IF left是float类型 OR right是float类型:
        IF left是int类型:
            left = builder->create_sitofp(left, floatType)
        END IF
        IF right是int类型:
            right = builder->create_sitofp(right, floatType)
        END IF
    END IF

    // 生成IR指令
    IF node->op == ADD:
        IF 是float类型:
            RETURN builder->create_fadd(left, right)
        ELSE:
            RETURN builder->create_iadd(left, right)
        END IF
    ELSE IF node->op == SUB:
        IF 是float类型:
            RETURN builder->create_fsub(left, right)
        ELSE:
            RETURN builder->create_isub(left, right)
        END IF
    END IF
END

算法: visitLVal(node, load)
输入: LValNode* node, bool load
输出: Value*

BEGIN
    info = symbolTable.lookup(node->ident)
    IF info == nullptr:
        报错: 未定义的变量
        RETURN nullptr
    END IF

    addr = info->value

    IF load == true:
        IF addr是Argument类型:
            RETURN addr  // 参数直接返回值
        END IF
        // 加载值
        loadType = addr->get_type()->get_pointer_element_type()
        RETURN builder->create_load(loadType, addr)
    ELSE:
        RETURN addr  // 返回地址
    END IF
END

算法: visitIfStmt(node)
输入: StmtNode* node
输出: 无

BEGIN
    // 创建基本块
    thenBB = BasicBlock::create(module, "", currentFunction)
    IF node->elseStmt存在:
        elseBB = BasicBlock::create(module, "", currentFunction)
    ELSE:
        elseBB = nullptr
    END IF
    mergeBB = BasicBlock::create(module, "", currentFunction)

    // 计算条件
    condVal = visitCond(node->cond)
    condVal = ensureInt1(condVal)  // 转换为i1类型

    // 条件跳转
    IF elseBB存在:
        builder->create_cond_br(condVal, thenBB, elseBB)
    ELSE:
        builder->create_cond_br(condVal, thenBB, mergeBB)
    END IF

    // 生成then分支
    currentBB = thenBB
    builder->set_insert_point(thenBB)
    visitStmt(node->thenStmt)
    IF currentBB没有终结指令:
        builder->create_br(mergeBB)
    END IF

    // 生成else分支
    IF elseBB存在 AND node->elseStmt存在:
        currentBB = elseBB
        builder->set_insert_point(elseBB)
        visitStmt(node->elseStmt)
        IF currentBB没有终结指令:
            builder->create_br(mergeBB)
        END IF
    END IF

    // 设置当前基本块为merge
    currentBB = mergeBB
    builder->set_insert_point(mergeBB)
END
\end{pseudocode}

\subsection{LLVM IR指令类型} % 对应源码文档 3.3.2

本编译器生成的IR指令类型包括:

\textbf{1. 终结指令(Terminator Instructions):}
\begin{itemize}
    \item \texttt{ret}: 返回指令
    \begin{itemize}
        \item \texttt{ret i32 \%val}: 返回值
        \item \texttt{ret void}: 无返回值返回
    \end{itemize}
    \item \texttt{br}: 跳转指令
    \begin{itemize}
        \item \texttt{br label \%bb}: 无条件跳转
        \item \texttt{br i1 \%cond, label \%if\_true, label \%if\_false}: 条件跳转
    \end{itemize}
\end{itemize}

\textbf{2. 二元运算指令(Binary Instructions):}
\begin{itemize}
    \item \textbf{整数运算}:
    \begin{itemize}
        \item \texttt{add}: 加法 (\texttt{\%result = add i32 \%a, \%b})
        \item \texttt{sub}: 减法 (\texttt{\%result = sub i32 \%a, \%b})
        \item \texttt{mul}: 乘法 (\texttt{\%result = mul i32 \%a, \%b})
        \item \texttt{sdiv}: 有符号除法 (\texttt{\%result = sdiv i32 \%a, \%b})
        \item \texttt{srem}: 有符号取模 (\texttt{\%result = srem i32 \%a, \%b})
    \end{itemize}
    \item \textbf{浮点运算}:
    \begin{itemize}
        \item \texttt{fadd}: 浮点加法 (\texttt{\%result = fadd float \%a, \%b})
        \item \texttt{fsub}: 浮点减法 (\texttt{\%result = fsub float \%a, \%b})
        \item \texttt{fmul}: 浮点乘法 (\texttt{\%result = fmul float \%a, \%b})
        \item \texttt{fdiv}: 浮点除法 (\texttt{\%result = fdiv float \%a, \%b})
    \end{itemize}
\end{itemize}

\textbf{3. 内存操作指令(Memory Instructions):}
\begin{itemize}
    \item \texttt{alloca}: 栈分配 (\texttt{\%ptr = alloca i32})
    \item \texttt{load}: 加载值 (\texttt{\%val = load i32, i32* \%ptr})
    \item \texttt{store}: 存储值 (\texttt{store i32 \%val, i32* \%ptr})
\end{itemize}

\textbf{4. 比较指令(Comparison Instructions):}
\begin{itemize}
    \item \textbf{整数比较}(返回i1类型):
    \begin{itemize}
        \item \texttt{icmp eq}: 相等 (\texttt{\%cmp = icmp eq i32 \%a, \%b})
        \item \texttt{icmp ne}: 不等 (\texttt{\%cmp = icmp ne i32 \%a, \%b})
        \item \texttt{icmp sgt}: 有符号大于 (\texttt{\%cmp = icmp sgt i32 \%a, \%b})
        \item \texttt{icmp sge}: 有符号大于等于 (\texttt{\%cmp = icmp sge i32 \%a, \%b})
        \item \texttt{icmp slt}: 有符号小于 (\texttt{\%cmp = icmp slt i32 \%a, \%b})
        \item \texttt{icmp sle}: 有符号小于等于 (\texttt{\%cmp = icmp sle i32 \%a, \%b})
    \end{itemize}
    \item \textbf{浮点比较}(返回i1类型):
    \begin{itemize}
        \item \texttt{fcmp oeq}: 有序相等 (\texttt{\%cmp = fcmp oeq float \%a, \%b})
        \item \texttt{fcmp one}: 有序不等 (\texttt{\%cmp = fcmp one float \%a, \%b})
        \item \texttt{fcmp ogt}: 有序大于 (\texttt{\%cmp = fcmp ogt float \%a, \%b})
        \item \texttt{fcmp oge}: 有序大于等于 (\texttt{\%cmp = fcmp oge float \%a, \%b})
        \item \texttt{fcmp olt}: 有序小于 (\texttt{\%cmp = fcmp olt float \%a, \%b})
        \item \texttt{fcmp ole}: 有序小于等于 (\texttt{\%cmp = fcmp ole float \%a, \%b})
    \end{itemize}
\end{itemize}

\textbf{5. 类型转换指令(Conversion Instructions):}
\begin{itemize}
    \item \texttt{zext}: 零扩展 (\texttt{\%ext = zext i1 \%val to i32})
    \item \texttt{sitofp}: 有符号整数转浮点 (\texttt{\%fp = sitofp i32 \%val to float})
    \item \texttt{fptosi}: 浮点转有符号整数 (\texttt{\%int = fptosi float \%val to i32})
\end{itemize}

\textbf{6. 函数调用指令(Call Instruction):}
\begin{itemize}
    \item \texttt{call}: 函数调用 (\texttt{\%result = call i32 @func(i32 \%arg)})
\end{itemize}

\textbf{7. Phi指令(Phi Instruction):}
\begin{itemize}
    \item \texttt{phi}: SSA形式的Phi节点 (\texttt{\%val = phi i32 [\%a, \%bb1], [\%b, \%bb2]})
\end{itemize}

\subsection{基本块和控制流图(CFG)} % 对应源码文档 3.3.3

\paragraph{基本块(Basic Block)概念:}
\begin{itemize}
    \item 基本块是最大化的线性指令序列,只有一个入口点和一个出口点
    \item 基本块以终结指令(\texttt{ret}或\texttt{br})结束
    \item 基本块之间通过跳转指令连接,形成控制流图(CFG)
\end{itemize}

\paragraph{基本块管理算法:}
\begin{pseudocode}
算法: 创建基本块
输入: 函数Function* func, 基本块名称string name
输出: BasicBlock*

BEGIN
    bb = BasicBlock::create(module, name, func)
    func->add_basic_block(bb)
    RETURN bb
END

算法: 设置当前基本块
输入: BasicBlock* bb
输出: 无

BEGIN
    currentBB = bb
    builder->set_insert_point(bb)  // 设置IRBuilder的插入点
END
\end{pseudocode}

\paragraph{控制流处理示例(if语句):}
\begin{verbatim}
IF语句的CFG结构:
    entryBB (入口基本块)
        |
        | (条件跳转)
        v
    +-------+-------+
    |               |
    v               v
thenBB          elseBB (可选)
    |               |
    +-------+-------+
            |
            v
        mergeBB (合并基本块)
\end{verbatim}

\subsection{符号表和作用域管理} % 对应源码文档 3.3.4

\paragraph{符号表结构:}
\begin{verbatim}
SymbolTable
├── scopes: vector<Scope*>  // 作用域栈
│   └── Scope
│       └── symbols: map<string, SymbolInfo>
│           └── SymbolInfo
│               ├── value: Value*      // IR中的Value指针
│               ├── type: Type*        // 类型
│               ├── isConst: bool      // 是否为常量
│               └── isGlobal: bool     // 是否为全局变量
\end{verbatim}

\paragraph{作用域管理算法:}
\begin{pseudocode}
算法: enterScope()
输出: 无

BEGIN
    scopes.push_back(make_shared<Scope>())
END

算法: exitScope()
输出: 无

BEGIN
    IF scopes.size() > 1:  // 保留全局作用域
        scopes.pop_back()
    END IF
END

算法: lookup(name)
输入: string name
输出: SymbolInfo*

BEGIN
    // 从内层作用域向外查找
    FOR i = scopes.size() - 1 TO 0:
        info = scopes[i]->lookup(name)
        IF info != nullptr:
            RETURN info
        END IF
    END FOR
    RETURN nullptr
END

算法: insert(name, value, type, isConst)
输入: string name, Value* value, Type* type, bool isConst
输出: bool (成功返回true)

BEGIN
    IF scopes.empty():
        RETURN false
    END IF

    info = SymbolInfo(value, type, isConst, isGlobalScope())
    RETURN scopes.back()->insert(name, info)
END
\end{pseudocode}

\subsection{全局变量和局部变量处理} % 对应源码文档 3.3.5

\paragraph{全局变量生成:}
\begin{pseudocode}
算法: visitConstDef (全局常量)
输入: ConstDefNode* node, BType bType
输出: 无

BEGIN
    IF symbolTable.isGlobalScope():
        type = bTypeToLLVMType(bType)

        // 计算常量初始值
        IF bType == INT:
            initVal = evalConstInt(node->initVal)
            init = ConstantInt::get(initVal, module)
        ELSE:
            initVal = evalConstFloat(node->initVal)
            init = ConstantFP::get(initVal, module)
        END IF

        // 创建全局变量
        gv = GlobalVariable::create(node->ident, module, type, true, init)
        symbolTable.insert(node->ident, gv, type, true)
    END IF
END
\end{pseudocode}

\paragraph{局部变量生成(SSA形式):}
\begin{pseudocode}
算法: visitVarDef (局部变量)
输入: VarDefNode* node, BType bType
输出: 无

BEGIN
    IF NOT symbolTable.isGlobalScope():
        type = bTypeToLLVMType(bType)

        // 创建alloca指令(在栈上分配空间)
        alloca = builder->create_alloca(type)

        // 如果有初始化值
        IF node->initVal存在:
            initVal = visitExp(node->initVal)
            builder->create_store(initVal, alloca)
        END IF

        symbolTable.insert(node->ident, alloca, type, false)
    END IF
END
\end{pseudocode}

\paragraph{变量访问:}
\begin{pseudocode}
算法: visitLVal(node, load)
输入: LValNode* node, bool load
输出: Value*

BEGIN
    info = symbolTable.lookup(node->ident)
    IF info == nullptr:
        报错: 未定义的变量
        RETURN nullptr
    END IF

    addr = info->value

    IF load == true:
        // 需要加载值
        IF addr是Argument类型:
            RETURN addr  // 函数参数直接返回值
        ELSE:
            // 生成load指令
            loadType = addr->get_type()->get_pointer_element_type()
            RETURN builder->create_load(loadType, addr)
        END IF
    ELSE:
        // 需要地址(用于赋值)
        RETURN addr
    END IF
END
\end{pseudocode}

\subsection{函数定义和调用} % 对应源码文档 3.3.6

\paragraph{函数定义生成:}
\begin{pseudocode}
算法: visitFuncDef(node)
输入: FuncDefNode* node
输出: 无

BEGIN
    // 1. 构建函数类型
    retType = bTypeToLLVMType(node->returnType)
    paramTypes = []
    FOR EACH param IN node->params:
        paramTypes.append(bTypeToLLVMType(param->bType))
    END FOR

    funcType = FunctionType::get(retType, paramTypes)

    // 2. 创建函数
    func = Function::create(funcType, node->ident, module)
    currentFunction = func
    builder->set_curFunc(func)

    // 3. 创建入口基本块
    entryBB = BasicBlock::create(module, node->ident + "_ENTRY", func)
    currentBB = entryBB
    builder->set_insert_point(entryBB)

    // 4. 进入函数作用域
    symbolTable.enterScope()

    // 5. 处理参数(为每个参数创建alloca并存储)
    argIt = func->arg_begin()
    FOR EACH param IN node->params:
        paramType = bTypeToLLVMType(param->bType)
        alloca = createLocalVariable(param->ident, paramType)
        builder->create_store(*argIt, alloca)
        symbolTable.insert(param->ident, alloca, paramType, false)
        argIt++
    END FOR

    // 6. 处理函数体
    visitBlock(node->block)

    // 7. 如果基本块没有终结指令,添加默认返回
    IF currentBB没有终结指令:
        IF retType是void:
            builder->create_void_ret()
        ELSE:
            builder->create_ret(ConstantInt::get(0, module))
        END IF
    END IF

    // 8. 退出函数作用域
    symbolTable.exitScope()
    currentFunction = nullptr
END
\end{pseudocode}

\paragraph{函数调用生成:}
\begin{pseudocode}
算法: visitUnaryExp (函数调用)
输入: UnaryExpNode* node (unaryType == FUNC_CALL)
输出: Value*

BEGIN
    // 1. 查找函数
    funcVal = symbolTable.getValue(node->funcName)
    IF funcVal == nullptr:
        报错: 未定义的函数
        RETURN nullptr
    END IF

    // 2. 准备参数
    args = []
    FOR EACH arg IN node->args:
        argVal = visitExp(arg)
        args.append(argVal)
    END FOR

    // 3. 创建调用指令
    RETURN builder->create_call(funcVal, args)
END
\end{pseudocode}

\subsection{类型系统和类型提升} % 对应源码文档 3.3.7

\paragraph{类型映射:}
\begin{verbatim}
BType -> LLVM Type:
    INT -> i32 (32位有符号整数)
    FLOAT -> float (32位浮点数)
    VOID -> void (无类型)
\end{verbatim}

\paragraph{类型提升规则:}
\begin{pseudocode}
算法: 类型提升(二元运算)
输入: Value* left, Value* right
输出: Value* left, Value* right (提升后的值)

BEGIN
    isFloat = left->get_type()->is_float_type() OR
              right->get_type()->is_float_type()

    IF isFloat:
        IF left是int32类型:
            left = builder->create_sitofp(left, floatType)
        END IF
        IF right是int32类型:
            right = builder->create_sitofp(right, floatType)
        END IF
    END IF

    RETURN left, right
END
\end{pseudocode}

\paragraph{类型转换指令生成:}
\begin{itemize}
    \item \texttt{sitofp}: 整数转浮点(用于混合类型运算)
    \item \texttt{fptosi}: 浮点转整数(用于赋值给整数变量)
    \item \texttt{zext}: 零扩展(i1转i32,用于条件判断)
\end{itemize}

\subsection{常量折叠优化} % 对应源码文档 3.3.8

\paragraph{常量求值算法:}
\begin{pseudocode}
算法: evalConstInt(node)
输入: ExpNode* node
输出: int

BEGIN
    IF node是NumberNode:
        RETURN node->intVal

    ELSE IF node是AddExpNode:
        IF node->left == nullptr:
            RETURN evalConstInt(node->right)
        ELSE:
            left = evalConstInt(node->left)
            right = evalConstInt(node->right)
            IF node->op == ADD:
                RETURN left + right
            ELSE IF node->op == SUB:
                RETURN left - right
            END IF
        END IF

    ELSE IF node是MulExpNode:
        IF node->left == nullptr:
            RETURN evalConstInt(node->right)
        ELSE:
            left = evalConstInt(node->left)
            right = evalConstInt(node->right)
            IF node->op == MUL:
                RETURN left * right
            ELSE IF node->op == DIV:
                RETURN left / right
            ELSE IF node->op == MOD:
                RETURN left % right
            END IF
        END IF

    ELSE IF node是UnaryExpNode:
        IF node->unaryType == PRIMARY:
            RETURN evalConstInt(node->primaryExp)
        ELSE IF node->unaryType == UNARY_OP:
            val = evalConstInt(node->unaryExp)
            IF node->unaryOp == PLUS:
                RETURN val
            ELSE IF node->unaryOp == MINUS:
                RETURN -val
            ELSE IF node->unaryOp == NOT:
                RETURN (val == 0 ? 1 : 0)
            END IF
        END IF

    ELSE IF node是PrimaryExpNode:
        IF node->primaryType == NUMBER:
            RETURN node->number->intVal
        ELSE IF node->primaryType == PAREN_EXP:
            RETURN evalConstInt(node->exp)
        END IF

    RETURN 0
END
\end{pseudocode}

\paragraph{常量折叠应用:}
\begin{itemize}
    \item 全局变量初始化时,如果初始值是常量表达式,直接计算并生成常量
    \item 避免生成不必要的运行时计算指令
\end{itemize}

\subsection{短路求值实现} % 对应源码文档 3.3.9

\paragraph{逻辑与(\&\&)短路求值:}
\begin{pseudocode}
算法: visitLAndExp(node)
输入: LAndExpNode* node
输出: Value* (i1类型)

BEGIN
    IF node->left == nullptr:
        // 只有右操作数(递归终止条件)
        RETURN visitEqExp(node->right)
    END IF

    // 创建基本块
    rhsBB = BasicBlock::create(module, "", currentFunction)
    mergeBB = BasicBlock::create(module, "", currentFunction)

    // 递归计算左操作数(左操作数也是LAndExp类型)
    leftVal = visitLAndExp(node->left)
    leftVal = ensureInt1(leftVal)  // 转换为i1类型
    leftBB = currentBB

    // 短路:左边为真则计算右边,否则跳到merge(整体为假)
    builder->create_cond_br(leftVal, rhsBB, mergeBB)

    // 计算右操作数(右操作数是EqExp类型)
    currentBB = rhsBB
    builder->set_insert_point(rhsBB)
    rightVal = visitEqExp(node->right)
    rightVal = ensureInt1(rightVal)
    builder->create_br(mergeBB)
    rightEndBB = currentBB

    // merge基本块
    currentBB = mergeBB
    builder->set_insert_point(mergeBB)

    // 创建phi节点:左边为假则整体为假,否则等于右边
    phi = PhiInst::create_phi(module->get_int1_type(), mergeBB)
    phi->add_phi_pair_operand(ConstantInt::get(false, module), leftBB)
    phi->add_phi_pair_operand(rightVal, rightEndBB)

    RETURN phi
END
\end{pseudocode}

\paragraph{逻辑或(||)短路求值:}
\begin{pseudocode}
算法: visitLOrExp(node)
输入: LOrExpNode* node
输出: Value* (i1类型)

BEGIN
    IF node->left == nullptr:
        // 只有右操作数(递归终止条件)
        RETURN visitLAndExp(node->right)
    END IF

    // 创建基本块
    rhsBB = BasicBlock::create(module, "", currentFunction)
    mergeBB = BasicBlock::create(module, "", currentFunction)

    // 递归计算左操作数(左操作数也是LOrExp类型)
    leftVal = visitLOrExp(node->left)
    leftVal = ensureInt1(leftVal)
    leftBB = currentBB

    // 短路:左边为真则跳到merge(整体为真),否则计算右边
    builder->create_cond_br(leftVal, mergeBB, rhsBB)

    // 计算右操作数(右操作数是LAndExp类型)
    currentBB = rhsBB
    builder->set_insert_point(rhsBB)
    rightVal = visitLAndExp(node->right)
    rightVal = ensureInt1(rightVal)
    builder->create_br(mergeBB)
    rightEndBB = currentBB

    // merge基本块
    currentBB = mergeBB
    builder->set_insert_point(mergeBB)

    // 创建phi节点:左边为真则整体为真,否则等于右边
    phi = PhiInst::create_phi(module->get_int1_type(), mergeBB)
    phi->add_phi_pair_operand(ConstantInt::get(true, module), leftBB)
    phi->add_phi_pair_operand(rightVal, rightEndBB)

    RETURN phi
END
\end{pseudocode}

\subsection{Phi节点和SSA形式} % 对应源码文档 3.3.10

\paragraph{Phi节点的作用:}
\begin{itemize}
    \item Phi节点用于SSA(Static Single Assignment)形式
    \item 当多个控制流路径汇合时,使用Phi节点合并不同路径的值
    \item 格式:\texttt{\%val = phi i32 [\%a, \%bb1], [\%b, \%bb2]}
    \begin{itemize}
        \item 如果从\texttt{\%bb1}到达,值为\texttt{\%a}
        \item 如果从\texttt{\%bb2}到达,值为\texttt{\%b}
    \end{itemize}
\end{itemize}

\paragraph{Phi节点生成示例(短路求值):}
\begin{verbatim}
; 逻辑与的CFG
entry:
    %left = icmp ne i32 %x, 0
    br i1 %left, label %rhs, label %merge

rhs:
    %right = icmp ne i32 %y, 0
    br label %merge

merge:
    %result = phi i1 [false, %entry], [%right, %rhs]
    ; 如果从entry到达(left为false),result为false
    ; 如果从rhs到达(left为true),result为right的值
\end{verbatim}

\subsection{IR生成完整流程} % 对应源码文档 3.3.11

\paragraph{主生成流程:}
\begin{pseudocode}
算法: generate(ast)
输入: CompUnitNode* ast
输出: 无

BEGIN
    // 1. 声明运行时库函数
    declareRuntimeFunctions()

    // 2. 遍历AST,生成IR
    visitCompUnit(ast)

    // 3. 设置打印名称(为所有Value分配名称)
    module->set_print_name()
END

算法: visitCompUnit(node)
输入: CompUnitNode* node
输出: 无

BEGIN
    // 先处理全局声明(全局变量和常量)
    FOR EACH decl IN node->decls:
        visitDecl(decl)
    END FOR

    // 再处理函数定义
    FOR EACH funcDef IN node->funcDefs:
        visitFuncDef(funcDef)
    END FOR
END
\end{pseudocode}

\paragraph{运行时库函数声明:}
\begin{lstlisting}[language=C++]
void IRGenerator::declareRuntimeFunctions() {
    std::vector<Type*> emptyParams;
    std::vector<Type*> intParam = {module->get_int32_type()};
    std::vector<Type*> intPtrParam = {module->get_int32_ptr_type()};
    std::vector<Type*> putArrayParams = {
        module->get_int32_type(),
        module->get_int32_ptr_type()
    };

    // int getint()
    FunctionType* getintType = FunctionType::get(
        module->get_int32_type(), emptyParams);
    Function::create(getintType, "getint", module);

    // int getch()
    FunctionType* getchType = FunctionType::get(
        module->get_int32_type(), emptyParams);
    Function::create(getchType, "getch", module);

    // int getarray(int*)
    FunctionType* getarrayType = FunctionType::get(
        module->get_int32_type(), intPtrParam);
    Function::create(getarrayType, "getarray", module);

    // void putint(int)
    FunctionType* putintType = FunctionType::get(
        module->get_void_type(), intParam);
    Function::create(putintType, "putint", module);

    // void putch(int)
    FunctionType* putchType = FunctionType::get(
        module->get_void_type(), intParam);
    Function::create(putchType, "putch", module);

    // void putarray(int, int*)
    FunctionType* putarrayType = FunctionType::get(
        module->get_void_type(), putArrayParams);
    Function::create(putarrayType, "putarray", module);

    // void starttime()
    FunctionType* starttimeType = FunctionType::get(
        module->get_void_type(), emptyParams);
    Function::create(starttimeType, "starttime", module);

    // void stoptime()
    FunctionType* stoptimeType = FunctionType::get(
        module->get_void_type(), emptyParams);
    Function::create(stoptimeType, "stoptime", module);
}
\end{lstlisting}

\paragraph{运行时库函数说明:}
\begin{itemize}
    \item \texttt{getint()}: 从标准输入读取一个整数
    \item \texttt{getch()}: 从标准输入读取一个字符
    \item \texttt{getarray(int*)}: 从标准输入读取一个整数数组
    \item \texttt{putint(int)}: 向标准输出写入一个整数
    \item \texttt{putch(int)}: 向标准输出写入一个字符
    \item \texttt{putarray(int, int*)}: 向标准输出写入一个整数数组
    \item \texttt{starttime()}: 开始计时
    \item \texttt{stoptime()}: 停止计时
\end{itemize}

\subsection{IR生成示例} % 对应源码文档 3.3.12

\paragraph{输入源代码:}
\begin{lstlisting}[language=C]
int a = 10;

int main() {
    int b = 20;
    if (a > 5) {
        b = a + 10;
    }
    return b;
}
\end{lstlisting}

\paragraph{生成的LLVM IR:}
\begin{verbatim}
@a = global i32 10

define i32 @main() {
entry:
    %0 = alloca i32
    %1 = alloca i32
    store i32 20, i32* %1
    %2 = load i32, i32* @a
    %3 = icmp sgt i32 %2, 5
    br i1 %3, label %then, label %merge

then:
    %4 = load i32, i32* @a
    %5 = add i32 %4, 10
    store i32 %5, i32* %1
    br label %merge

merge:
    %6 = load i32, i32* %1
    ret i32 %6
}
\end{verbatim}

\paragraph{IR说明:}
\begin{enumerate}
    \item \texttt{@a = global i32 10}: 全局变量定义
    \item \texttt{\%0 = alloca i32}: 为局部变量分配栈空间(未使用)
    \item \texttt{\%1 = alloca i32}: 为变量\texttt{b}分配栈空间
    \item \texttt{store i32 20, i32* \%1}: 初始化\texttt{b = 20}
    \item \texttt{\%2 = load i32, i32* @a}: 加载全局变量\texttt{a}
    \item \texttt{\%3 = icmp sgt i32 \%2, 5}: 比较\texttt{a > 5}
    \item \texttt{br i1 \%3, label \%then, label \%merge}: 条件跳转
    \item \texttt{then}基本块:计算\texttt{b = a + 10}
    \item \texttt{merge}基本块:加载\texttt{b}并返回
\end{enumerate}

\subsection{代码实现细节} % 对应源码文档 3.3.13

\paragraph{IRGenerator类核心成员:}
\begin{lstlisting}[language=C++]
class IRGenerator {
private:
    Module* module;              // 当前模块
    IRBuilder* builder;          // IR构建器
    SymbolTable symbolTable;     // 符号表
    Function* currentFunction;   // 当前函数
    BasicBlock* currentBB;       // 当前基本块

    // 临时存储常量值(用于常量折叠)
    int tmpIntVal;
    float tmpFloatVal;
    bool tmpIsFloat;
    bool isConstExpr;

    // 用于短路求值的基本块
    BasicBlock* trueBB;
    BasicBlock* falseBB;
};
\end{lstlisting}

\paragraph{构造函数和初始化:}
\begin{lstlisting}[language=C++]
IRGenerator::IRGenerator(const std::string& sourceFileName);
\end{lstlisting}

\textbf{功能说明:}创建IR模块和IRBuilder,初始化成员变量,声明运行时库函数。

\paragraph{类型转换辅助函数:}
\begin{lstlisting}[language=C++]
Type* IRGenerator::bTypeToLLVMType(BType bType);  // BType转换为LLVM Type
Value* IRGenerator::ensureInt1(Value* val);       // 确保值为i1类型(布尔值)
\end{lstlisting}

\textbf{功能说明:}
\begin{itemize}
    \item \texttt{bTypeToLLVMType()}: 将BType(INT/FLOAT/VOID)映射到对应的LLVM类型(i32/float/void)
    \item \texttt{ensureInt1()}: 如果值是i32类型,则与0比较转换为i1类型;如果已经是i1类型则直接返回
\end{itemize}

\paragraph{常量求值:}
\begin{lstlisting}[language=C++]
int IRGenerator::evalConstInt(std::shared_ptr<ExpNode> node);
\end{lstlisting}

\textbf{功能说明:}递归计算常量表达式的整数值,支持数字字面量、加法表达式、乘法表达式和一元表达式。用于常量折叠优化。

\paragraph{表达式访问(类型提升):}
\begin{lstlisting}[language=C++]
Value* IRGenerator::visitAddExp(std::shared_ptr<AddExpNode> node);
\end{lstlisting}

\textbf{功能说明:}访问加法表达式节点,递归处理左操作数,处理类型提升(如果有一个操作数是float,则将int转换为float),生成对应的IR指令(fadd/iadd或fsub/isub)。

\paragraph{左值访问(加载/地址):}
\begin{lstlisting}[language=C++]
Value* IRGenerator::visitLVal(std::shared_ptr<LValNode> node, bool load);
\end{lstlisting}

\textbf{功能说明:}访问左值节点。如果\texttt{load=true},则从符号表查找变量并生成load指令加载值(函数参数直接返回值);如果\texttt{load=false},则返回变量的地址(用于赋值)。

\paragraph{if语句实现(控制流):}
\begin{lstlisting}[language=C++]
void IRGenerator::visitIfStmt(std::shared_ptr<StmtNode> node);
\end{lstlisting}

\textbf{功能说明:}生成if语句的IR。创建thenBB、elseBB(如果存在)和mergeBB基本块,计算条件值并生成条件跳转,分别生成then和else分支的IR,最后设置当前基本块为mergeBB。

\subsection{设计思想总结} % 对应源码文档 3.3.14

\paragraph{核心设计原则:}
\begin{enumerate}
    \item \textbf{访问者模式}:为每种AST节点类型定义专门的访问函数,实现关注点分离
    \item \textbf{SSA形式}:使用alloca/store/load模式实现局部变量,符合LLVM IR的SSA要求
    \item \textbf{基本块管理}:每个函数有入口基本块,控制流语句创建新基本块,使用Phi节点合并控制流
    \item \textbf{符号表作用域}:使用作用域栈管理变量可见性,支持嵌套作用域
    \item \textbf{类型系统}:统一的类型映射和类型提升规则,确保类型安全
    \item \textbf{常量折叠}:编译时计算常量表达式,减少运行时开销
    \item \textbf{短路求值}:逻辑运算符使用控制流实现短路,提高效率
    \item \textbf{错误处理}:在IR生成过程中检查未定义变量、类型不匹配等错误
\end{enumerate}

\paragraph{IR生成示例:}

输入AST:
\begin{verbatim}
CompUnit
├── VarDecl (int a = 10)
└── FuncDef (int main() { return 0; })
\end{verbatim}

生成的LLVM IR:
\begin{verbatim}
@a = global i32 10

define i32 @main() {
entry:
    ret i32 0
}
\end{verbatim}

\subsection{符号表实现细节} % 对应源码文档 3.4

\subsubsection{符号表数据结构} % 对应源码文档 3.4.1

\paragraph{SymbolInfo结构:}
\begin{lstlisting}[language=C++]
struct SymbolInfo {
    Value* value;           // IR中的Value指针
    Type* type;            // 类型
    bool isConst;           // 是否为常量
    bool isGlobal;          // 是否为全局变量

    SymbolInfo() : value(nullptr), type(nullptr),
                   isConst(false), isGlobal(false) {}
    SymbolInfo(Value* v, Type* t, bool c, bool g)
        : value(v), type(t), isConst(c), isGlobal(g) {}
};
\end{lstlisting}

\paragraph{Scope类实现:}
\begin{lstlisting}[language=C++]
class Scope {
public:
    std::map<std::string, SymbolInfo> symbols;

    SymbolInfo* lookup(const std::string& name);  // 在当前作用域中查找符号
    bool insert(const std::string& name, const SymbolInfo& info);  // 在当前作用域中插入符号
};
\end{lstlisting}

\textbf{功能说明:}
\begin{itemize}
    \item \texttt{lookup()}: 在当前作用域中查找符号,找到返回SymbolInfo指针,否则返回nullptr
    \item \texttt{insert()}: 在当前作用域中插入符号,如果符号已存在则返回false(重复定义),否则返回true
\end{itemize}

\paragraph{SymbolTable类实现:}
\begin{lstlisting}[language=C++]
class SymbolTable {
private:
    std::vector<std::shared_ptr<Scope>> scopes;  // 作用域栈

public:
    SymbolTable();  // 构造函数,创建全局作用域
    void enterScope();  // 进入新的作用域
    void exitScope();  // 退出当前作用域(保留全局作用域)
    bool isGlobalScope() const;  // 判断当前是否在全局作用域
    SymbolInfo* lookup(const std::string& name);  // 从内层作用域向外查找符号
    bool insert(const std::string& name, Value* value, Type* type, bool isConst);  // 在当前作用域中插入符号
};
\end{lstlisting}

\textbf{功能说明:}
\begin{itemize}
    \item \texttt{enterScope()}: 创建新的作用域并压入作用域栈
    \item \texttt{exitScope()}: 弹出当前作用域(但保留全局作用域)
    \item \texttt{isGlobalScope()}: 判断当前作用域栈大小是否为1(即全局作用域)
    \item \texttt{lookup()}: 从内层作用域向外层查找符号,实现作用域嵌套查找
    \item \texttt{insert()}: 在当前作用域(栈顶)中插入符号
\end{itemize}

\subsection{IRBuilder实现细节} % 对应源码文档 3.5

\subsubsection{IRBuilder核心接口} % 对应源码文档 3.5.1

\paragraph{IRBuilder类结构:}
\begin{lstlisting}[language=C++]
class IRBuilder {
private:
    BasicBlock *BB_;        // 当前基本块
    Module *m_;            // 模块
    Function *curfunc;     // 当前函数

public:
    IRBuilder(BasicBlock *bb, Module *m);

    // 设置插入点
    void set_insert_point(BasicBlock *bb);

    // 整数运算指令
    BinaryInst* create_iadd(Value *lhs, Value *rhs);
    BinaryInst* create_isub(Value *lhs, Value *rhs);
    BinaryInst* create_imul(Value *lhs, Value *rhs);
    BinaryInst* create_isdiv(Value *lhs, Value *rhs);

    // 浮点运算指令
    BinaryInst* create_fadd(Value *lhs, Value *rhs);
    BinaryInst* create_fsub(Value *lhs, Value *rhs);
    BinaryInst* create_fmul(Value *lhs, Value *rhs);
    BinaryInst* create_fdiv(Value *lhs, Value *rhs);

    // 比较指令
    CmpInst* create_icmp_eq(Value *lhs, Value *rhs);
    CmpInst* create_icmp_ne(Value *lhs, Value *rhs);
    CmpInst* create_icmp_sgt(Value *lhs, Value *rhs);
    CmpInst* create_icmp_sge(Value *lhs, Value *rhs);
    CmpInst* create_icmp_slt(Value *lhs, Value *rhs);
    CmpInst* create_icmp_sle(Value *lhs, Value *rhs);
    CmpInst* create_fcmp_oeq(Value *lhs, Value *rhs);
    CmpInst* create_fcmp_ogt(Value *lhs, Value *rhs);
    // ... 其他比较指令

    // 内存操作指令
    AllocaInst* create_alloca(Type *ty);
    LoadInst* create_load(Type *ty, Value *ptr);
    StoreInst* create_store(Value *val, Value *ptr);

    // 控制流指令
    BranchInst* create_br(BasicBlock *if_true);
    BranchInst* create_cond_br(Value *cond, BasicBlock *if_true, BasicBlock *if_false);
    ReturnInst* create_ret(Value *val);
    ReturnInst* create_void_ret();

    // 类型转换指令
    ZextInst* create_zext(Value *val, Type *ty);
    SiToFpInst* create_sitofp(Value *val, Type *ty);
    FpToSiInst* create_fptosi(Value *val, Type *ty);

    // 函数调用指令
    CallInst* create_call(Value *func, std::vector<Value *> args);
};
\end{lstlisting}

\textbf{功能说明:}
IRBuilder提供创建各种LLVM IR指令的便捷接口,所有方法都在当前基本块(BB\_)中创建指令:
\begin{itemize}
    \item \textbf{整数运算}:\texttt{create\_iadd}, \texttt{create\_isub}, \texttt{create\_imul}, \texttt{create\_isdiv}等
    \item \textbf{浮点运算}:\texttt{create\_fadd}, \texttt{create\_fsub}, \texttt{create\_fmul}, \texttt{create\_fdiv}等
    \item \textbf{比较指令}:\texttt{create\_icmp\_*}(整数比较),\texttt{create\_fcmp\_*}(浮点比较)
    \item \textbf{内存操作}:\texttt{create\_alloca}(栈分配),\texttt{create\_load}(加载),\texttt{create\_store}(存储)
    \item \textbf{控制流}:\texttt{create\_br}(无条件跳转),\texttt{create\_cond\_br}(条件跳转),\texttt{create\_ret}(返回)
    \item \textbf{类型转换}:\texttt{create\_zext}(零扩展),\texttt{create\_sitofp}(整数转浮点),\texttt{create\_fptosi}(浮点转整数)
    \item \textbf{函数调用}:\texttt{create\_call}(函数调用)
\end{itemize}

\subsection{源程序编译步骤(完整流程)} % 对应源码文档 3.6

\paragraph{完整编译流程:}

\begin{enumerate}
    \item \textbf{词法分析}:
    \begin{lstlisting}[language=C++]
SLRLexer lexer;
auto tokens = lexer.analyze(sourceCode);
    \end{lstlisting}

    \item \textbf{语法分析}:
    \begin{lstlisting}[language=C++]
SLRParser parser;
bool success = parser.parse(tokens);
auto ast = parser.getAST();
    \end{lstlisting}

    \item \textbf{IR生成}:
    \begin{lstlisting}[language=C++]
IRGenerator generator("source.sy");
generator.generate(ast);
std::string ir = generator.print();
    \end{lstlisting}
\end{enumerate}

\paragraph{命令行使用:}
\begin{verbatim}
./compiler -i source.sy    # 完整编译(生成LLVM IR)
\end{verbatim}

\clearpage
\section{总结}

本编译器实现了完整的编译流程:
\begin{enumerate}
    \item \textbf{词法分析}:将源代码转换为Token序列
    \item \textbf{语法分析}:将Token序列解析为AST
    \item \textbf{IR生成}:遍历AST生成LLVM IR中间代码
\end{enumerate}

\paragraph{关键特性:}
\begin{itemize}
    \item 支持两种词法分析器(DFA状态机、NFA→DFA)
    \item 支持两种语法分析器(递归下降、SLR)
    \item 完整的AST节点体系
    \item 符号表作用域管理
    \item 类型系统和类型提升
    \item 常量折叠优化
    \item 短路求值
    \item 控制流图生成
\end{itemize}

\clearpage
\section{小组成员分工与心得体会}

\subsection{成员一:\NameA\ (学号:\IdA)}

\paragraph{主要分工:}
\WorkA

\paragraph{心得体会:}
\ReflectionA

\subsection{成员二:\NameB\ (学号:\IdB)}

\paragraph{主要分工:}
\WorkB

\paragraph{心得体会:}
\ReflectionB

\subsection{成员三:\NameC\ (学号:\IdC)}

\paragraph{主要分工:}
\WorkC

\paragraph{心得体会:}
\ReflectionC

\subsection{成员四:\NameD\ (学号:\IdD)}

\paragraph{主要分工:}
\WorkD

\paragraph{心得体会:}
\ReflectionD

\end{document}
