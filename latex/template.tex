% ======================================================================
% 导言区 (Preamble)
% ======================================================================
\documentclass[11pt, a4paper]{article}

% --- 页面边距设置 ---
\usepackage[a4paper, top=2.5cm, bottom=2.5cm, left=2cm, right=2cm]{geometry}

% --- 字体与语言设置 (核心) ---
\usepackage{fontspec} % 允许使用系统字体
\usepackage{ctex}

% --- TOC 深度设置 (新增或修改) ---
\setcounter{tocdepth}{3} % 目录深度到 subsubsection
\setcounter{secnumdepth}{3} % 编号深度到 subsubsection

% 修正非英语语言环境下的列表标签
\usepackage{enumitem}
\setlist[itemize]{label=-}

% --- 其他常用宏包 ---
\usepackage{amsmath} % 数学公式
\usepackage{booktabs} % 用于 \addlinespace,改善表格线条
\usepackage{graphicx} % 图像支持 (封面占位符需要)
\usepackage{hyperref} % 超链接 (始终最后加载)
\usepackage{indentfirst}
\hypersetup{
    colorlinks=true,
    linkcolor=blue,
    filecolor=magenta,
    urlcolor=cyan,
    pdftitle={编译器构造报告},
}

% 定义用于伪代码和代码块的样式
\usepackage{listings}
% 使用 listings 宏包定义一个用于伪代码的环境
\lstset{
    basicstyle=\small\ttfamily,
    columns=fullflexible,
    breaklines=true,
    frame=single,
    framerule=0pt,
}
\lstnewenvironment{pseudocode}[1][]{
    \lstset{
        language=Mathematica, % 使用一个不常用的语言来禁用语法高亮
        #1
    }
}{}

% ======================================================================
% 文档区 (Document)
% ======================================================================
\begin{document}

% ----------------------------------------------------------------------
% 封面页 (Title Page)
% ----------------------------------------------------------------------
\begin{titlepage}

    % personal_info.tex
% 定义小组及个人信息变量

% --- 公共信息 (假设大家是同一个专业/班级) ---
\newcommand{\StudentMajor}{软件工程}
\newcommand{\StudentGrade}{2023级}
\newcommand{\StudentClass}{软工2班}

% --- 小组成员信息 (4人) ---
% 成员 1
\newcommand{\NameA}{陈添硕}
\newcommand{\IdA}{3023244225}
\newcommand{\WorkA}{语法分析器设计与实现}
\newcommand{\ReflectionA}{在实现语法分析器的过程中,深入理解了自顶向下和自底向上两种分析方法的区别与联系。通过实现递归下降分析器,掌握了如何将语法规则转化为递归函数,理解了向前看(lookahead)在解决选择冲突中的重要作用。在构建SLR分析表时,通过计算FIRST集和FOLLOW集,深刻体会到了形式化方法在编译器设计中的严谨性。语法分析器的实现让我认识到,良好的设计需要充分考虑语法的特点和实际应用场景。}

% 成员 2
\newcommand{\NameB}{李昊蓬}
\newcommand{\IdB}{3023244162}
\newcommand{\WorkB}{IR生成器设计和实现}
\newcommand{\ReflectionB}{IR生成是编译器中最核心也最复杂的部分。通过实现访问者模式遍历AST并生成LLVM IR,深入理解了中间代码生成的过程。在处理类型提升、常量折叠、短路求值等特性时,深刻体会到了编译器优化的复杂性。符号表的作用域管理是一个重要挑战,通过实现作用域栈成功解决了变量可见性问题。控制流图的生成和Phi节点的使用让我理解了SSA形式的重要性。这次经历让我对编译器的后端有了更深入的认识。}

% 成员 3
\newcommand{\NameC}{许英帅}
\newcommand{\IdC}{3023244221}
\newcommand{\WorkC}{词法分析器设计和实现}
\newcommand{\ReflectionC}{通过实现词法分析器,深入理解了词法分析的基本原理。在实现DFA状态机的过程中,掌握了如何将正则表达式转化为确定有限自动机,理解了状态转换表的设计方法。关键字识别、标识符扫描、数字解析和注释处理等细节的实现让我认识到词法分析的复杂性。通过实现NFA到DFA的转换,进一步理解了形式语言理论在实际应用中的价值。词法分析器作为编译器的第一道关卡,其正确性对整个编译过程至关重要。}

% 成员 4
\newcommand{\NameD}{谢雨航}
\newcommand{\IdD}{3023244222}
\newcommand{\WorkD}{测试与调试,文档编写}
\newcommand{\ReflectionD}{在测试与调试过程中,学会了如何系统地验证编译器的正确性。通过设计测试用例覆盖各种边界情况,发现了许多隐藏的bug。调试过程让我深刻理解了编译器各个模块之间的依赖关系,学会了如何定位问题所在。文档编写工作让我对整个编译器的架构有了全面的认识,通过整理和总结,不仅帮助了团队协作,也加深了自己对编译器原理的理解。这次经历让我认识到,测试和文档是软件开发中不可或缺的重要环节。}

    \centering % 整体居中

    % 1. 校徽 (使用占位符)
    \vspace*{2cm} % 距离顶部一些空间

    \includegraphics[width=0.5\linewidth]{logo.png}

    \vfill % 自动填充垂直空间

    % 2. 报告名称
    {\Huge C--编译器源码结构和功能详细说明}

    \vspace{1cm}

    {\Large 编译器构造报告}

    \vfill % 自动填充垂直空间

	% 3. 小组信息展示 (四人信息格式)
    % 调整了 makebox 的宽度 (从5cm改为7cm) 以容纳"姓名+学号"
    \begin{tabular}{rl}
        \textbf{专    业:} & \underline{\makebox[7cm][c]{\StudentMajor}} \\
        \addlinespace[10pt]
        \textbf{年    级:} & \underline{\makebox[7cm][c]{\StudentGrade}} \\
        \addlinespace[10pt]
        \textbf{班    级:} & \underline{\makebox[7cm][c]{\StudentClass}} \\
        \addlinespace[10pt]

        % 成员列表
        \textbf{组    员:} & \underline{\makebox[7cm][c]{\NameA \quad (\IdA)}} \\
        \addlinespace[8pt] % 调整行间距
                            & \underline{\makebox[7cm][c]{\NameB \quad (\IdB)}} \\
        \addlinespace[8pt]
                            & \underline{\makebox[7cm][c]{\NameC \quad (\IdC)}} \\
        \addlinespace[8pt]
                            & \underline{\makebox[7cm][c]{\NameD \quad (\IdD)}} \\
    \end{tabular}

    \vspace{3cm} % 距离底部一些空间
\end{titlepage}

% ----------------------------------------------------------------------
% 报告正文 (Main Body) - 替换为源码说明内容
% ----------------------------------------------------------------------
\clearpage

\section*{目录} % 为目录添加标题
\addcontentsline{toc}{section}{目录}
\tableofcontents % 添加目录
\clearpage

% --- 编译器源码结构和功能详细说明 ---

\section{词法分析器}

\subsection{算法描述}

本项目实现了两种词法分析器:

\subsubsection{方法一:基于DFA状态机的词法分析器 (\texttt{Lexer})} % 已修改为 \subsubsection

\paragraph{算法流程:}
\begin{enumerate}
    \item \textbf{初始化阶段}:
    \begin{itemize}
        \item 初始化关键字表(\texttt{int}, \texttt{void}, \texttt{return}, \texttt{const}, \texttt{float}, \texttt{if}, \texttt{else})
        \item 设置当前位置指针 $\texttt{pos} = 0$,行号 $\texttt{line} = 1$,列号 $\texttt{column} = 1$
    \end{itemize}
    \item \textbf{主分析循环} (\texttt{tokenize()} 函数):
    \begin{pseudocode}
WHILE pos < source.length():
    跳过空白字符 (空格、制表符、换行符)
    跳过注释 (// 行注释 或 /* */ 块注释)

    IF 当前位置字符是字母或下划线:
        调用 scanIdentifier() 扫描标识符或关键字
    ELSE IF 当前位置字符是数字:
        调用 scanNumber() 扫描整数或浮点数
    ELSE IF 当前位置字符是界符 ( ( ) { } ; , ):
        调用 scanSeparator() 扫描界符
    ELSE IF 当前位置字符是运算符 ( + - * / \% = < > ! & | ):
        调用 scanOperator() 扫描运算符
    ELSE:
        创建ERROR token

    将扫描到的token添加到tokens列表
END WHILE

添加EOF token
    \end{pseudocode}

    \item \textbf{DFA状态转换}:
    \begin{itemize}
        \item \texttt{START}: 初始状态
        \item \texttt{IN\_ID}: 标识符状态(字母/下划线开头,后跟字母/数字/下划线)
        \item \texttt{IN\_NUM}: 整数状态(数字序列)
        \item \texttt{IN\_FLOAT}: 浮点数状态(数字.数字)
        \item \texttt{IN\_ASSIGN}: =状态(可能是=或==)
        \item \texttt{IN\_LT}: <状态(可能是<或<=)
        \item \texttt{IN\_GT}: >状态(可能是>或>=)
        \item \texttt{IN\_NOT}: !状态(需要!=)
        \item \texttt{IN\_AND}: \&状态(需要\&\&)
        \item \texttt{IN\_OR}: $|$状态(需要$||$)
        \item \texttt{IN\_COMMENT\_LINE}: 行注释状态
        \item \texttt{IN\_COMMENT\_BLOCK}: 块注释状态
    \end{itemize}
\end{enumerate}

\subsubsection{方法二:基于NFA→DFA转换的词法分析器 (\texttt{SLRLexer})} % 已修改为 \subsubsection

\paragraph{算法流程:}
\begin{enumerate}
    \item \textbf{构建NFA} (\texttt{buildDFA()} 函数):
    \begin{itemize}
        \item 为关键字、标识符、整数、浮点数、运算符、界符分别构建NFA
        \item 使用epsilon转换合并所有NFA为一个组合NFA
        \item 通过子集构造法(Subset Construction)将NFA转换为DFA
        \item 使用DFA最小化算法优化DFA
    \end{itemize}
    \item \textbf{最长匹配算法} (\texttt{analyze()} 函数):
    \begin{pseudocode}
初始化: currentState = dfa.start, startPos = pos

WHILE pos < source.length():
    跳过空白字符和注释

    currentState = dfa.start
    startPos = pos
    lastAcceptState = nullptr
    lastAcceptPos = -1
    currentPos = pos

    // 最长匹配
    WHILE currentPos < source.length():
         char c = sourceCode[currentPos]
         IF currentState有c的转换:
             currentState = currentState.transitions[c]
             currentPos++
             IF currentState是接受状态:
                 lastAcceptState = currentState
                 lastAcceptPos = currentPos
         ELSE:
             BREAK
    END WHILE

    IF lastAcceptState != nullptr AND lastAcceptPos > startPos:
        创建token: Token(lastAcceptState->acceptType,
                        sourceCode.substr(startPos, lastAcceptPos - startPos))
         pos = lastAcceptPos
    ELSE:
        创建ERROR token
         pos++
END WHILE

添加EOF token
    \end{pseudocode}
\end{enumerate}

\subsection{输出格式说明}

\paragraph{Token输出格式:}
\texttt{[单词符号] TAB <[类型],[属性]>}

\paragraph{类型分类:}
\begin{itemize}
    \item \textbf{KW (关键字)}: \texttt{int(1)}, \texttt{void(2)}, \texttt{return(3)}, \texttt{const(4)}, \texttt{float(6)}, \texttt{if(7)}, \texttt{else(8)}
    \item \textbf{OP (运算符)}: \texttt{+(30)}, \texttt{-(31)}, \texttt{*(32)}, \texttt{/(33)}, \texttt{\%(34)}, \texttt{=(35)}, \texttt{>(36)}, \texttt{<(37)}, \texttt{==(38)}, \texttt{<=(39)}, \texttt{>=(40)}, \texttt{!=(41)}, \texttt{\&\&(42)}, \texttt{||(43)}, \texttt{!(44)}
    \item \textbf{SE (界符)}: \texttt{((50)}, \texttt{)(51)}, \texttt{\{(52)}, \texttt{\}(53)}, \texttt{;(54)}, \texttt{,(55)}
    \item \textbf{IDN (标识符)}: \texttt{<IDN,标识符名称>}
    \item \textbf{INT (整数)}: \texttt{<INT,整数值>}
    \item \textbf{FLOAT (浮点数)}: \texttt{<FLOAT,浮点数值>}
    \item \textbf{EOF (文件结束)}: \texttt{\$}
\end{itemize}

\paragraph{示例输出:}
\begin{verbatim}
int	<KW,1>
main	<IDN,main>
(	<SE,50>
)	<SE,51>
{	<SE,52>
return	<KW,3>
10	<INT,10>
;	<SE,54>
}	<SE,53>
\end{verbatim}

\subsection{源程序编译步骤}

\paragraph{使用词法分析器的编译步骤:}
\begin{enumerate}
    \item \textbf{加载源文件}:
    \begin{verbatim}
Lexer lexer;
lexer.loadFromFile("source.sy");
    \end{verbatim}
    \item \textbf{执行词法分析}:
    \begin{verbatim}
lexer.tokenize();
    \end{verbatim}
    \item \textbf{获取Token序列}:
    \begin{verbatim}
const auto& tokens = lexer.getTokens();
    \end{verbatim}
    \item \textbf{输出结果}:
    \begin{verbatim}
lexer.printTokens();  // 打印到标准输出
// 或
std::string tokenStr = lexer.getTokensString();  // 获取字符串
    \end{verbatim}
\end{enumerate}

\paragraph{命令行使用:}
\begin{verbatim}
./compiler -l source.sy    # 仅词法分析
\end{verbatim}

\hrule

\section{语法分析器}

\subsection{算法描述}

本项目实现了两种语法分析器:

\subsubsection{方法一:递归下降语法分析器 (\texttt{Parser})} % 已修改为 \subsubsection

\paragraph{算法特点:}
\begin{itemize}
    \item 为每个非终结符编写一个递归函数
    \item 通过向前看(lookahead)解决选择冲突
    \item 自顶向下构建AST
\end{itemize}

\paragraph{主要分析函数:}
\begin{itemize}
    \item \texttt{parseCompUnit()}: 解析编译单元
    \item \texttt{parseConstDecl()}: 解析常量声明
    \item \texttt{parseVarDecl()}: 解析变量声明
    \item \texttt{parseFuncDef()}: 解析函数定义
    \item \texttt{parseStmt()}: 解析语句
    \item \texttt{parseExp()}: 解析表达式(通过\texttt{parseAddExp})
    \item \texttt{parseCond()}: 解析条件表达式
\end{itemize}

\paragraph{算法流程示例(\texttt{parseAddExp}):}
\begin{pseudocode}
parseAddExp():
    left = parseMulExp()

    WHILE 当前token是 + 或 -:
        保存运算符op
        前进token
        right = parseMulExp()
        创建新的AddExpNode,left作为左子树,right作为右子树
        left = 新节点
    END WHILE

    RETURN left
\end{pseudocode}

\subsubsection{方法二:SLR语法分析器 (\texttt{SLRParser})} % 已修改为 \subsubsection

\paragraph{算法流程:}
\begin{enumerate}
    \item \textbf{初始化文法} (\texttt{initGrammar()}):
    \begin{itemize}
        \item 定义81个产生式规则
        \item 识别终结符和非终结符集合
    \end{itemize}
    \item \textbf{计算FIRST集} (\texttt{computeFirst()}):
    \begin{pseudocode}
FOR 每个终结符t:
    FIRST(t) = {t}

WHILE FIRST集发生变化:
    FOR 每个产生式 A -> α:
        IF α = ε:
            FIRST(A) += {ε}
        ELSE:
            FOR 每个符号X in α:
                FIRST(A) += FIRST(X) - {ε}
                IF ε NOT IN FIRST(X):
                    BREAK
                IF 到达α的末尾:
                    FIRST(A) += {ε}
END WHILE
    \end{pseudocode}
    \item \textbf{计算FOLLOW集} (\texttt{computeFollow()}):
    \begin{pseudocode}
FOLLOW(开始符号) = {$}

WHILE FOLLOW集发生变化:
    FOR 每个产生式 A -> αBβ:
        IF β存在:
            FOLLOW(B) += FIRST(β) - {ε}
            IF ε IN FIRST(β):
                FOLLOW(B) += FOLLOW(A)
        ELSE:
            FOLLOW(B) += FOLLOW(A)
END WHILE
    \end{pseudocode}
    \item \textbf{构建LR(0)项目集族} (\texttt{buildCollection()}):
    \begin{pseudocode}
I0 = closure({S' -> ·Program})
canonicalCollection = {I0}

WHILE 有新的项目集产生:
    FOR 每个项目集I:
        FOR 每个符号X (终结符或非终结符):
            J = goto(I, X)
            IF J非空 AND J不在canonicalCollection中:
                将J添加到canonicalCollection
                记录转换: (I, X) -> J
    END WHILE
    \end{pseudocode}
    \item \textbf{构建SLR分析表} (\texttt{buildTable()}):
    \begin{pseudocode}
FOR 每个项目集Ii:
    FOR 每个项目 [A -> α·aβ] in Ii:
        J = goto(Ii, a)
        action[i, a] = shift j

    FOR 每个项目 [A -> α·] in Ii:
        IF A != S':
            FOR 每个a IN FOLLOW(A):
                action[i, a] = reduce A -> α
        ELSE:
            action[i, $] = accept

    FOR 每个非终结符A:
        IF goto(Ii, A) = Ij:
            goto[i, A] = j
    \end{pseudocode}
    \item \textbf{SLR分析算法} (\texttt{parse()}):
    \begin{pseudocode}
初始化: stateStack = [0], valueStack = []
ip = 0  // 输入指针

WHILE true:
    s = stateStack.top()
    a = tokens[ip]  // 当前输入符号

    IF action[s, a] = shift t:
        stateStack.push(t)
        valueStack.push(语义值)
        ip++

    ELSE IF action[s, a] = reduce A -> β:
        从栈中弹出|β|个状态和语义值
        t = stateStack.top()
        stateStack.push(goto[t, A])
        valueStack.push(reduce(产生式ID, 语义值列表))

    ELSE IF action[s, a] = accept:
        RETURN success

    ELSE:
        RETURN error
END WHILE
    \end{pseudocode}
\end{enumerate}

\subsection{SLR分析表结构}

\paragraph{分析表包含两部分:}
\begin{enumerate}
    \item \textbf{ACTION表}:$\texttt{actionTable[state, terminal] -> Action}$
    \item \textbf{GOTO表}:$\texttt{gotoTable[state, nonTerminal] -> state}$
\end{enumerate}

\paragraph{冲突解决:}
\begin{itemize}
    \item 移进-归约冲突:优先移进(\texttt{shift-reduce conflict resolution})
\end{itemize}

\paragraph{示例分析表片段:}
\begin{verbatim}
State 0:
  action[0, "int"] = shift 5
  action[0, "const"] = shift 6
  goto[0, "Program"] = 1
  goto[0, "compUnit"] = 2
...
\end{verbatim}

\subsection{输出格式说明}

\paragraph{递归下降分析器输出格式:}
\begin{verbatim}
步骤号	TAB	Token值	TAB	动作
1	int	move
2	main	move
...
13	$#$	accept
\end{verbatim}

\subsection{源程序编译步骤}

\paragraph{命令行使用:}
\begin{verbatim}
./compiler -p source.sy    # 词法+语法分析
\end{verbatim}

\hrule

\section{语法树存储与遍历及IR生成}

\subsection{语法树存储结构}

\paragraph{AST节点层次结构(简化):}
\begin{verbatim}
CompUnitNode (根节点)
├── decls: vector<DeclNode*>
│   └── ...
└── funcDefs: vector<FuncDefNode*>
    └── FuncDefNode
        └── block: BlockNode*
\end{verbatim}

\paragraph{表达式节点层次(简化):}
\begin{verbatim}
ExpNode (基类)
├── AddExpNode / MulExpNode
...
└── NumberNode
\end{verbatim}

\paragraph{存储方式:}
\begin{itemize}
    \item 使用 \texttt{std::shared\_ptr} 智能指针管理节点生命周期
\end{itemize}

\subsection{语法树遍历算法伪代码}

\paragraph{访问者模式遍历AST(部分示例):}
\begin{pseudocode}
算法: visitCompUnit(node)
输入: CompUnitNode* node
输出: 无

BEGIN
    // 先处理全局声明
    FOR EACH decl IN node->decls:
        visitDecl(decl)
    END FOR

    // 再处理函数定义
    FOR EACH funcDef IN node->funcDefs:
        visitFuncDef(funcDef)
    END FOR
END

算法: visitAddExp(node)
...
\end{pseudocode}

\subsection{调用中端的过程的算法设计思想}

\paragraph{IR生成器的设计思想:}
\begin{enumerate}
    \item \textbf{访问者模式}:为每种AST节点类型定义访问函数,递归遍历AST。
    \item \textbf{符号表管理}:使用作用域栈管理局部变量和全局变量的查找与插入。
    \item \textbf{基本块管理}:为控制流语句(如 \texttt{if}, \texttt{while})创建新的基本块。
    \item \textbf{类型系统}:将 \texttt{BType} 映射到 \texttt{LLVM Type}。
    \item \textbf{常量折叠}:在IR生成前预先计算常量表达式的值。
    \item \textbf{短路求值}(逻辑与/或):通过条件跳转和 \texttt{Phi} 节点实现短路逻辑。
\end{enumerate}

\paragraph{IR生成示例:}
\begin{verbatim}
@a = global i32 10

define i32 @main() {
entry:
    ret i32 0
}
\end{verbatim}

\subsection{源程序编译步骤}

\paragraph{命令行使用:}
\begin{verbatim}
./compiler -i source.sy    # 完整编译(生成LLVM IR)
\end{verbatim}

\hrule

\section{总结}

本编译器实现了完整的编译流程:
\begin{itemize}
    \item \textbf{词法分析}:将源代码转换为 \texttt{Token} 序列
    \item \textbf{语法分析}:将 \texttt{Token} 序列解析为 \texttt{AST}
    \item \textbf{IR生成}:遍历 \texttt{AST} 生成 \texttt{LLVM IR} 中间代码
\end{itemize}
关键特性:
\begin{itemize}
    \item 支持两种词法分析器(DFA状态机、NFA→DFA)
    \item 支持两种语法分析器(递归下降、SLR)
    \item 完整的 \texttt{AST} 节点体系
    \item 符号表作用域管理
    \item 类型系统和类型提升
    \item 常量折叠优化
    \item 短路求值
    \item 控制流图生成
\end{itemize}

\end{document}
