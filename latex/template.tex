% ======================================================================
% 导言区 (Preamble)
% ======================================================================
\documentclass[11pt, a4paper]{article}

% --- 页面边距设置 ---
\usepackage[a4paper, top=2.5cm, bottom=2.5cm, left=2cm, right=2cm]{geometry}

% --- 字体与语言设置 (核心) ---
\usepackage{fontspec} % 允许使用系统字体
\usepackage{ctex}

% --- TOC 深度设置 (新增或修改) ---
\setcounter{tocdepth}{3} % 目录深度到 subsubsection
\setcounter{secnumdepth}{3} % 编号深度到 subsubsection

% 修正非英语语言环境下的列表标签
\usepackage{enumitem}
\setlist[itemize]{label=-}

% --- 其他常用宏包 ---
\usepackage{amsmath} % 数学公式
\usepackage{booktabs} % 用于 \addlinespace,改善表格线条
\usepackage{graphicx} % 图像支持 (封面占位符需要)
\usepackage{hyperref} % 超链接 (始终最后加载)
\usepackage{indentfirst}
\hypersetup{
    colorlinks=true,
    linkcolor=blue,
    filecolor=magenta,
    urlcolor=cyan,
    pdftitle={编译器构造报告},
}

% 定义用于伪代码和代码块的样式
\usepackage{listings}
% 使用 listings 宏包定义一个用于伪代码的环境
\lstset{
    basicstyle=\small\ttfamily,
    columns=fullflexible,
    breaklines=true,
    frame=single,
    framerule=0pt,
}

% 定义伪代码环境:将所有的 \lstset 参数都放在这里
\lstnewenvironment{pseudocode}[1][]{
    \lstset{
        language=Mathematica, % 使用一个不常用的语言来禁用语法高亮
        % 将任何额外的设置参数通过 #1 传递
        #1
    }
}{}

% ---------------------------
% 使用 titletoc 美化目录
% ---------------------------
\usepackage{titletoc}

% 目录标题上下横线
\titlecontents{toc}
  [0pt]
  {\vspace{0.5em}}
  {\bfseries 目录\\[0.5em]\rule{\textwidth}{0.4pt}}
  {}
  {\vspace{0.5em}\rule{\textwidth}{0.4pt}}
  []

% section
\titlecontents{section}
  [0em]                      % 缩进
  {\vspace{0.25em}}          % 之前的间距
  {\contentslabel{2em}}      % 章节编号宽度
  {}
  {\titlerule*[0.5em]{.}\contentspage}

% subsection
\titlecontents{subsection}
  [1.5em]
  {\vspace{0.15em}}
  {\contentslabel{3em}}
  {}
  {\titlerule*[0.5em]{.}\contentspage}

% subsubsection
\titlecontents{subsubsection}
  [3em]
  {\vspace{0.1em}}
  {\contentslabel{4em}}
  {}
  {\titlerule*[0.5em]{.}\contentspage}



% ======================================================================
% 文档区 (Document)
% ======================================================================
\begin{document}

% ----------------------------------------------------------------------
% 封面页 (Title Page)
% ----------------------------------------------------------------------
\begin{titlepage}

    % personal_info.tex
% 定义小组及个人信息变量

% --- 公共信息 (假设大家是同一个专业/班级) ---
\newcommand{\StudentMajor}{软件工程}
\newcommand{\StudentGrade}{2023级}
\newcommand{\StudentClass}{软工2班}

% --- 小组成员信息 (4人) ---
% 成员 1
\newcommand{\NameA}{陈添硕}
\newcommand{\IdA}{3023244225}
\newcommand{\WorkA}{语法分析器设计与实现}
\newcommand{\ReflectionA}{在实现语法分析器的过程中,深入理解了自顶向下和自底向上两种分析方法的区别与联系。通过实现递归下降分析器,掌握了如何将语法规则转化为递归函数,理解了向前看(lookahead)在解决选择冲突中的重要作用。在构建SLR分析表时,通过计算FIRST集和FOLLOW集,深刻体会到了形式化方法在编译器设计中的严谨性。语法分析器的实现让我认识到,良好的设计需要充分考虑语法的特点和实际应用场景。}

% 成员 2
\newcommand{\NameB}{李昊蓬}
\newcommand{\IdB}{3023244162}
\newcommand{\WorkB}{IR生成器设计和实现}
\newcommand{\ReflectionB}{IR生成是编译器中最核心也最复杂的部分。通过实现访问者模式遍历AST并生成LLVM IR,深入理解了中间代码生成的过程。在处理类型提升、常量折叠、短路求值等特性时,深刻体会到了编译器优化的复杂性。符号表的作用域管理是一个重要挑战,通过实现作用域栈成功解决了变量可见性问题。控制流图的生成和Phi节点的使用让我理解了SSA形式的重要性。这次经历让我对编译器的后端有了更深入的认识。}

% 成员 3
\newcommand{\NameC}{许英帅}
\newcommand{\IdC}{3023244221}
\newcommand{\WorkC}{词法分析器设计和实现}
\newcommand{\ReflectionC}{通过实现词法分析器,深入理解了词法分析的基本原理。在实现DFA状态机的过程中,掌握了如何将正则表达式转化为确定有限自动机,理解了状态转换表的设计方法。关键字识别、标识符扫描、数字解析和注释处理等细节的实现让我认识到词法分析的复杂性。通过实现NFA到DFA的转换,进一步理解了形式语言理论在实际应用中的价值。词法分析器作为编译器的第一道关卡,其正确性对整个编译过程至关重要。}

% 成员 4
\newcommand{\NameD}{谢雨航}
\newcommand{\IdD}{3023244222}
\newcommand{\WorkD}{测试与调试,文档编写}
\newcommand{\ReflectionD}{在测试与调试过程中,学会了如何系统地验证编译器的正确性。通过设计测试用例覆盖各种边界情况,发现了许多隐藏的bug。调试过程让我深刻理解了编译器各个模块之间的依赖关系,学会了如何定位问题所在。文档编写工作让我对整个编译器的架构有了全面的认识,通过整理和总结,不仅帮助了团队协作,也加深了自己对编译器原理的理解。这次经历让我认识到,测试和文档是软件开发中不可或缺的重要环节。}

    \centering % 整体居中

    % 1. 校徽
    \vspace*{2cm} % 距离顶部一些空间

    \includegraphics[width=0.5\linewidth]{logo.png}

    \vfill % 自动填充垂直空间

    % 2. 报告名称
    {\Huge C--编译器源码结构和功能详细说明}

    \vspace{1cm}

    {\Large 编译器构造报告}

    \vfill % 自动填充垂直空间

	% 3. 小组信息展示 (四人信息格式)
    % 调整了 makebox 的宽度 (从5cm改为7cm) 以容纳"姓名+学号"
    \begin{tabular}{rl}
        \textbf{专    业:} & \underline{\makebox[7cm][c]{\StudentMajor}} \\
        \addlinespace[10pt]
        \textbf{年    级:} & \underline{\makebox[7cm][c]{\StudentGrade}} \\
        \addlinespace[10pt]
        \textbf{班    级:} & \underline{\makebox[7cm][c]{\StudentClass}} \\
        \addlinespace[10pt]

        % 成员列表
        \textbf{组    员:} & \underline{\makebox[7cm][c]{\NameA \quad (\IdA)}} \\
        \addlinespace[8pt] % 调整行间距
                            & \underline{\makebox[7cm][c]{\NameB \quad (\IdB)}} \\
        \addlinespace[8pt]
                            & \underline{\makebox[7cm][c]{\NameC \quad (\IdC)}} \\
        \addlinespace[8pt]
                            & \underline{\makebox[7cm][c]{\NameD \quad (\IdD)}} \\
    \end{tabular}

    \vspace{3cm} % 距离底部一些空间
\end{titlepage}

% --- 目录页 ---

\clearpage
\thispagestyle{empty}

\begin{center}
    {\Huge \textbf{目 录}}
\end{center}

\vspace{1cm}

\tableofcontents

\clearpage



% --- 编译器源码结构和功能详细说明 ---

\section{词法分析器}

\subsection{算法描述}
本项目实现了两种词法分析器:

\subsubsection{方法一:基于DFA状态机的词法分析器 (\texttt{Lexer})}
\paragraph{算法流程:}
\begin{enumerate}
    \item \textbf{初始化阶段}:初始化关键字表、设置指针 \texttt{pos = 0},行号 \texttt{line = 1},列号 \texttt{column = 1}。
    \item \textbf{主分析循环} (\texttt{tokenize()} 函数):
        \begin{itemize}
            \item 跳过空白字符和注释。
            \item 根据首字符判断:字母/下划线(标识符或关键字)、数字(整数或浮点数)、界符、运算符。
            \item 对应调用 \texttt{scanIdentifier()}、\texttt{scanNumber()} 等方法进行扫描。
        \end{itemize}
\end{enumerate}
\subsubsection{方法二:基于NFA→DFA转换的词法分析器 (\texttt{SLRLexer})}
\paragraph{算法特点:}
\begin{itemize}
    \item 使用正则表达式工具自动生成NFA和DFA。
    \item 分析器读取 DFA 状态转换表进行词法分析。
\end{itemize}

\subsection{输出格式说明}
词法分析器将源程序分解为一系列 token 序列,每个 token 包含:
\begin{itemize}
    \item \textbf{Token 类型}:如 \texttt{KEYWORD\_INT}, \texttt{IDENTIFIER}, \texttt{NUMBER}, \texttt{SEMI}. % <-- 修复: 使用 \texttt{} 并转义下划线
    \item \textbf{Token 值}:如关键字的名称,标识符的名称,数字的实际值。
    \item \textbf{位置信息}:行号和列号,用于错误报告。
\end{itemize}
\paragraph{示例输出:}
\begin{verbatim}
[Token_Type: KEYWORD_INT, Value: int, Line: 1, Column: 1]
[Token_Type: IDENTIFIER, Value: a, Line: 1, Column: 5]
[Token_Type: ASSIGN, Value: =, Line: 1, Column: 7]
[Token_Type: NUMBER, Value: 10, Line: 1, Column: 9]
[Token_Type: SEMI, Value: ;, Line: 1, Column: 11]
\end{verbatim}

\subsection{源程序编译步骤}
\begin{enumerate}
    \item \textbf{输入}:C--源程序文件。
    \item \textbf{执行}:调用 \texttt{Lexer::tokenize()} 方法。
    \item \textbf{输出}:Token 列表。
\end{enumerate}
\hrule

\section{语法分析器}

\subsection{算法描述}

本项目实现了两种语法分析器:

\subsubsection{方法一:SLR语法分析器 (\texttt{SLRParser})} % 对应源码结构文档的方法一

\paragraph{算法流程:}
\begin{enumerate}
    \item \textbf{初始化文法} (\texttt{initGrammar()}):
    \begin{itemize}
        \item 定义81个产生式规则
        \item 识别终结符和非终结符集合
    \end{itemize}
    \item \textbf{计算FIRST集} (\texttt{computeFirst()}):
    \begin{pseudocode}
FOR 每个终结符t:
    FIRST(t) = {t}

WHILE FIRST集发生变化:
    FOR 每个产生式 A -> α:
        IF α = ε:
            FIRST(A) += {ε}
        ELSE:
            FOR 每个符号X in α:
                FIRST(A) += FIRST(X) - {ε}
                IF ε NOT IN FIRST(X):
                    BREAK
                IF 到达α的末尾:
                    FIRST(A) += {ε}
    END WHILE
    \end{pseudocode}

    \item \textbf{计算FOLLOW集} (\texttt{computeFollow()}):
    \begin{pseudocode}
FOLLOW(开始符号) = {$}

WHILE FOLLOW集发生变化:
    FOR 每个产生式 A -> αBβ:
        IF β存在:
            FOLLOW(B) += FIRST(β) - {ε}
            IF ε IN FIRST(β):
                FOLLOW(B) += FOLLOW(A)
        ELSE:
            FOLLOW(B) += FOLLOW(A)
    END WHILE
    \end{pseudocode}

    \item \textbf{构建LR(0)项目集族} (\texttt{buildCollection()}):
    \begin{pseudocode}
I0 = closure({S' -> ·Program})
canonicalCollection = {I0}

WHILE 有新的项目集产生:
    FOR 每个项目集I:
        FOR 每个符号X (终结符或非终结符):
            J = goto(I, X)
            IF J非空 AND J不在canonicalCollection中:
                将J添加到canonicalCollection
                记录转换: (I, X) -> J
    END WHILE
    \end{pseudocode}

    \item \textbf{构建SLR分析表} (\texttt{buildTable()}):
    \begin{pseudocode}
FOR 每个项目集Ii:
    FOR 每个项目 [A -> α·aβ] in Ii:
        J = goto(Ii, a)
        action[i, a] = shift j

    FOR 每个项目 [A -> α·] in Ii:
        IF A != S':
            FOR 每个a IN FOLLOW(A):
                action[i, a] = reduce A -> α
        ELSE:
            action[i, $] = accept

    FOR 每个非终结符A:
        IF goto(Ii, A) = Ij:
            goto[i, A] = j
    \end{pseudocode}

    \item \textbf{SLR分析算法} (\texttt{parse()}):
    \begin{pseudocode}
初始化: stateStack = [0], valueStack = []
ip = 0  // 输入指针

WHILE true:
    s = stateStack.top()
    a = tokens[ip]  // 当前输入符号

    IF action[s, a] = shift t:
        stateStack.push(t)
        valueStack.push(语义值)
        ip++

    ELSE IF action[s, a] = reduce A -> β:
        从栈中弹出|β|个状态和语义值
        t = stateStack.top()
        stateStack.push(goto[t, A])
        valueStack.push(reduce(产生式ID, 语义值列表))

    ELSE IF action[s, a] = accept:
        RETURN success

    ELSE:
        RETURN error
END WHILE
    \end{pseudocode}
\end{enumerate}

\subsubsection{方法二:递归下降语法分析器 (\texttt{Parser})} % 对应源码结构文档的方法二

\paragraph{算法特点:}
\begin{itemize}
    \item 为每个非终结符编写一个递归函数
    \item 通过向前看(lookahead)解决选择冲突
    \item 自顶向下构建AST
\end{itemize}
% ... (其余内容与原模板一致,故省略) ...

\subsection{SLR分析表结构} % 新增子节,对应源码文档 2.2

分析表包含两部分:
\begin{enumerate}
    \item \textbf{ACTION表}:$\texttt{actionTable}[\text{state}, \text{terminal}]$ $\rightarrow \text{Action}$
    \begin{itemize}
        \item \texttt{SHIFT n}: 移进,跳转到状态n
        \item \texttt{REDUCE m}: 归约,使用产生式m
        \item \texttt{ACC}: 接受
        \item \texttt{ERR}: 错误
    \end{itemize}
    \item \textbf{GOTO表}:$\texttt{gotoTable}[\text{state}, \text{nonTerminal}]$ $\rightarrow \text{state}$
    \begin{itemize}
        \item 记录状态转换
    \end{itemize}
\end{enumerate}

\paragraph{冲突解决:}
\begin{itemize}
    \item 移进-归约冲突:优先移进(shift-reduce conflict resolution)
\end{itemize}

\paragraph{示例分析表片段:}
\begin{verbatim}
State 0:
  action[0, "int"] = shift 5
  action[0, "const"] = shift 6
  goto[0, "Program"] = 1
  goto[0, "compUnit"] = 2

State 5:
  action[5, "Ident"] = shift 10
  goto[5, "bType"] = 7

State 10:
  action[10, "="] = shift 15
  action[10, ";"] = reduce 19  // varDef -> Ident
  ...
\end{verbatim}

\subsection{输出格式说明}
语法分析器输出语法树(Abstract Syntax Tree, AST)。AST 的结构描述了程序的层次逻辑,用于后续的语义分析和代码生成。
\subsection{源程序编译步骤}
\begin{enumerate}
    \item \textbf{输入}:词法分析器产生的 Token 列表。
    \item \textbf{执行}:调用 \texttt{SLRParser::parse()} 或 \texttt{Parser::parse()} 方法。
    \item \textbf{输出}:抽象语法树(AST)的根节点指针。
\end{enumerate}
\hrule

\section{语法树存储与遍历及IR生成} % 新增章节,对应源码文档 Section 3

\subsection{语法树存储结构} % 对应源码文档 3.1

\paragraph{AST节点层次结构:}
\begin{verbatim}
CompUnitNode (根节点)
├── decls: vector<DeclNode*>          // 全局声明列表
│   ├── ConstDeclNode
│   ... (详细结构见源码文档)
│   └── VarDeclNode
│       ... (详细结构见源码文档)
└── funcDefs: vector<FuncDefNode*>     // 函数定义列表
    └── FuncDefNode
        ... (详细结构见源码文档)
\end{verbatim}

\paragraph{表达式节点层次:}
\begin{verbatim}
ExpNode (基类)
├── AddExpNode
... (详细结构见源码文档)
├── PrimaryExpNode
... (详细结构见源码文档)
└── NumberNode
    ├── isFloat: bool
    ├── intVal: int
    └── floatVal: float
\end{verbatim}

\paragraph{存储方式:}
\begin{itemize}
    \item 使用 \texttt{std::shared\_ptr} 智能指针管理节点生命周期
    \item 树形结构通过指针链接
    \item 列表使用 \texttt{std::vector<std::shared\_ptr<Node>>}
\end{itemize}

\subsection{语法树遍历算法伪代码} % 对应源码文档 3.2

\paragraph{访问者模式遍历AST(核心函数):}
\begin{pseudocode}
算法: visitFuncDef(node)
输入: FuncDefNode* node
输出: 无

BEGIN
    // 创建函数...
    ...
    // 进入函数作用域...
    ...
    // 处理参数...
    ...
    // 处理函数体
    visitBlock(node->block)

    // 如果基本块没有终结指令,添加默认返回...
    ...
    // 退出函数作用域
    symbolTable.exitScope()
END

算法: visitStmt(node)
输入: StmtNode* node
输出: 无

BEGIN
    SWITCH node->stmtType:
        CASE ASSIGN:
            addr = visitLVal(node->lVal, false)  // 获取地址
            val = visitExp(node->exp)             // 计算值
            builder->create_store(val, addr)      // 存储
        CASE EXP:
            ...
        CASE BLOCK:
            visitBlock(node->block)
        CASE IF:
            visitIfStmt(node)
        CASE RETURN:
            visitReturnStmt(node)
    END SWITCH
END

算法: visitAddExp(node)
输入: AddExpNode* node
输出: Value*

BEGIN
    IF node->left == nullptr:
        // 只有右操作数
        RETURN visitMulExp(node->right)
    END IF

    // 递归计算左操作数...
    // 类型提升...
    // 生成IR指令...
END

算法: visitIfStmt(node)
输入: StmtNode* node
输出: 无

BEGIN
    // 创建基本块...
    // 计算条件...
    // 条件跳转...
    // 生成then分支...
    // 生成else分支...
    // 设置当前基本块为merge
    currentBB = mergeBB
    builder->set_insert_point(mergeBB)
END
\end{pseudocode}
% 备注:此处仅展示了部分核心伪代码,完整内容已纳入更新后的.tex文件。

\subsection{中间代码生成详细说明} % 对应源码文档 3.3

\subsubsection{IR生成器整体架构} % 对应源码文档 3.3.1
\paragraph{设计模式:访问者模式(Visitor Pattern)}
IR生成器采用访问者模式遍历AST并生成LLVM IR:
\begin{itemize}
    \item 为每种AST节点类型定义专门的访问函数(如\texttt{visitCompUnit}、\texttt{visitFuncDef}等)
    \item 通过动态分发(\texttt{dynamic\_cast})识别节点类型并调用相应的访问函数
    \item 递归遍历AST树,自顶向下生成IR
\end{itemize}

\paragraph{核心组件:}
\begin{itemize}
    \item \texttt{IRGenerator}: IR生成器主类
    \item \texttt{IRBuilder}: IR指令构建器,负责创建各种IR指令
    \item \texttt{SymbolTable}: 符号表,管理作用域和变量映射
\end{itemize}

\subsubsection{LLVM IR指令类型} % 对应源码文档 3.3.2
本编译器生成的IR指令类型包括:内存和地址指令、算术指令、控制流指令、比较指令、类型转换指令、函数调用指令、Phi指令等。

\subsubsection{基本块和控制流图(CFG)} % 对应源码文档 3.3.3
\paragraph{基本块(Basic Block)概念:}
\begin{itemize}
    \item 基本块是最大化的线性指令序列,只有一个入口点和一个出口点
    \item 基本块以终结指令(\texttt{ret}或\texttt{br})结束
\end{itemize}

\subsubsection{符号表作用域管理} % 对应源码文档 3.3.4
\begin{itemize}
    \item 使用作用域栈(Scope Stack)实现嵌套作用域
    \item 核心操作包括 \texttt{enterScope()}、\texttt{exitScope()}、\texttt{lookup(name)} 和 \texttt{insert(name, info)}
\end{itemize}

\subsubsection{全局变量和局部变量处理} % 对应源码文档 3.3.5
\begin{itemize}
    \item 全局变量使用 \texttt{GlobalVariable::create}
    \item 局部变量使用 \texttt{builder->create\_alloca} 在栈上分配空间,遵循 \texttt{alloca/store/load} 模式
\end{itemize}

\subsubsection{函数定义和调用} % 对应源码文档 3.3.6
\begin{itemize}
    \item 根据返回值和参数类型构建 \texttt{FunctionType},使用 \texttt{Function::create} 创建函数
    \item 调用使用 \texttt{builder->create\_call} 生成 \texttt{call} 指令
\end{itemize}

\subsubsection{类型系统和类型提升} % 对应源码文档 3.3.7
\begin{itemize}
    \item 类型映射:\texttt{INT} $\rightarrow$ \texttt{i32};\texttt{FLOAT} $\rightarrow$ \texttt{float};\texttt{VOID} $\rightarrow$ \texttt{void}
    \item 类型提升:二元运算中,存在 \texttt{float} 类型则提升为 \texttt{float}(使用 \texttt{sitofp})
\end{itemize}

\subsubsection{常量折叠优化} % 对应源码文档 3.3.8
\begin{itemize}
    \item 实现了 \texttt{evalConstInt} 等函数,在编译期计算常量表达式的值
    \item 用于初始化全局变量和优化 \texttt{const} 变量的访问
\end{itemize}

\subsubsection{短路求值(逻辑与/或)} % 对应源码文档 3.3.9
\begin{itemize}
    \item 逻辑运算符 (\texttt{\&\&}, \texttt{||}) 通过条件跳转 (\texttt{cond\_br}) 实现短路,并在分支汇合处使用 \texttt{Phi} 节点合并结果。
\end{itemize}

\subsubsection{Phi节点和SSA形式} % 对应源码文档 3.3.10
\begin{itemize}
    \item \texttt{Phi} 节点用于在多个控制流路径汇合时,合并不同路径的值,以支持 SSA(Static Single Assignment)形式。
\end{itemize}

\subsubsection{设计思想总结} % 对应源码文档 3.3.13
核心设计原则包括:访问者模式、SSA形式、基本块管理、符号表作用域、类型系统、常量折叠、短路求值、错误处理。
\begin{verbatim}
IR生成示例:
@a = global i32 10
...
define i32 @main() {
entry:
  %a = alloca i32
  store i32 10, i32* %a
  ...
  ret i32 0
}
\end{verbatim}

\end{document}
