% personal_info.tex
% 定义小组及个人信息变量

% --- 公共信息 (假设大家是同一个专业/班级) ---
\newcommand{\StudentMajor}{软件工程}
\newcommand{\StudentGrade}{2023级}
\newcommand{\StudentClass}{软工2班}

% --- 小组成员信息 (4人) ---
% 成员 1
\newcommand{\NameA}{陈添硕}
\newcommand{\IdA}{3023244225}
\newcommand{\WorkA}{语法分析器设计与实现}
\newcommand{\ReflectionA}{在实现语法分析器的过程中,深入理解了自顶向下和自底向上两种分析方法的区别与联系。通过实现递归下降分析器,掌握了如何将语法规则转化为递归函数,理解了向前看(lookahead)在解决选择冲突中的重要作用。在构建SLR分析表时,通过计算FIRST集和FOLLOW集,深刻体会到了形式化方法在编译器设计中的严谨性。语法分析器的实现让我认识到,良好的设计需要充分考虑语法的特点和实际应用场景。}

% 成员 2
\newcommand{\NameB}{李昊蓬}
\newcommand{\IdB}{3023244162}
\newcommand{\WorkB}{IR生成器设计和实现}
\newcommand{\ReflectionB}{IR生成是编译器中最核心也最复杂的部分。通过实现访问者模式遍历AST并生成LLVM IR,深入理解了中间代码生成的过程。在处理类型提升、常量折叠、短路求值等特性时,深刻体会到了编译器优化的复杂性。符号表的作用域管理是一个重要挑战,通过实现作用域栈成功解决了变量可见性问题。控制流图的生成和Phi节点的使用让我理解了SSA形式的重要性。这次经历让我对编译器的后端有了更深入的认识。}

% 成员 3
\newcommand{\NameC}{许英帅}
\newcommand{\IdC}{3023244221}
\newcommand{\WorkC}{词法分析器设计和实现}
\newcommand{\ReflectionC}{通过实现词法分析器,深入理解了词法分析的基本原理。在实现DFA状态机的过程中,掌握了如何将正则表达式转化为确定有限自动机,理解了状态转换表的设计方法。关键字识别、标识符扫描、数字解析和注释处理等细节的实现让我认识到词法分析的复杂性。通过实现NFA到DFA的转换,进一步理解了形式语言理论在实际应用中的价值。词法分析器作为编译器的第一道关卡,其正确性对整个编译过程至关重要。}

% 成员 4
\newcommand{\NameD}{谢雨航}
\newcommand{\IdD}{3023244222}
\newcommand{\WorkD}{测试与调试,文档编写}
\newcommand{\ReflectionD}{在测试与调试过程中,学会了如何系统地验证编译器的正确性。通过设计测试用例覆盖各种边界情况,发现了许多隐藏的bug。调试过程让我深刻理解了编译器各个模块之间的依赖关系,学会了如何定位问题所在。文档编写工作让我对整个编译器的架构有了全面的认识,通过整理和总结,不仅帮助了团队协作,也加深了自己对编译器原理的理解。这次经历让我认识到,测试和文档是软件开发中不可或缺的重要环节。}
